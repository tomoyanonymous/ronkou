\chapter{序論}\label{ux5e8fux8ad6}

筆者は卒業制作として、音響遅延線メモリーというコンピューターの初期に使われていた記憶装置を題材にした音響装置作品の制作を行った。\\
音響遅延線という装置の概要については本論で述べることにするが、一連の制作においてはすでに淘汰され使われなくなったメディア装置、その中でも特に記憶装置を取り上げ、現代の技術を取り込み別のメディア装置として再生させるという作品制作のプロセスを取った。

\section{動機と問題意識}\label{ux52d5ux6a5fux3068ux554fux984cux610fux8b58}

今日、デジタルネイティブという言葉が存在するように我々が何か技術を使うときにアナログ、デジタルと言う境目を意識することは少なくなったように思える。また、それに伴い「保存する」ことと「通信する」事の境界は曖昧になりつつ有る。

例えば、MicrosoftのWordに代表される文書作成ソフトをWebブラウザ上で動作させるようにしたGoogle
Docs\footnote{Googleの提供する、ウェブブラウザ上で動作する文書作成ソフト。Wordなどの既存のフォーマットを扱うこともできる。\url{https://www.google.co.jp/intl/ja/docs/about/}}では書き込んだデータはリアルタイムでサーバに送信され自動で保存され、さらに共同編集をしている場合自分以外の書き込みも逐次反映されていく。\\
ここで、手紙の紙=保存メディアの発展形であったはずのGoogle
Docsはもはや手紙という行為そのもの=通信にもなっている。そしてこれらを使うとき、石版や紙のような媒体の物質性は意識されず、人間同士のやり取りだけが意識される。\\
このような時代においてそもそも情報を保存する、伝えるとは一体どういうことなのだろうか?

情報を保存するというのは、その前提に伝えるということがあって始まるのだが、通信技術の発達と計算機の性能向上によって、わざわざ一度保存せずとも、速く遠くに伝えることができるようになってしまった。それでは何を目的として何の情報を保存する必要があるのだろうか?

その疑問に迫るために筆者はコンピューター最初期に使われた「通信し続けることで保存する記憶装置」をである音響遅延線を復元させるという行為を行った。

\section{メディアアートをどう考察するべきか?}\label{ux30e1ux30c7ux30a3ux30a2ux30a2ux30fcux30c8ux3092ux3069ux3046ux8003ux5bdfux3059ux308bux3079ux304dux304b}

筆者は「メディアアート」と呼ばれる領域に興味を持ち、自分の制作以外にもメディアアートと呼ばれるモノのサウンドプログラムの開発などを行なってきた。\\
そのため本論考においても「メディアアート」としての作品についての考察を行いたいと考えている。\\
すなわち「この作品はメディアアートの文脈の中でどういう位置づけにあるか?」という問いに答えなければいけないが、これには様々な困難がつきまとう。そもそもメディアアートとは何か?という問いに対して統一された見解のようなものは今も昔も存在しているとは言えないからである。\\
本論考に於いては日本におけるメディアアートの歴史をまとめる数少ない試みの内の一つである、2014年に発行された馬定延「日本メディアアート史」において用いられた「作品が作られる舞台となった施設やイベント、社会的状況を中心にメディアアートの歴史を考察する」というアプローチを参考にし、敢えて先行作品例との比較などでの文脈付けを行わずに考察する。

\section{メディア考古学という学問アプローチ}\label{ux30e1ux30c7ux30a3ux30a2ux8003ux53e4ux5b66ux3068ux3044ux3046ux5b66ux554fux30a2ux30d7ux30edux30fcux30c1}

先行作品例などではない考察をするための議論の軸として、一つのヒントとなる学問のアプローチが有る。今回の制作では「すでに淘汰されたメディア装置を研究し、別の形のメディア装置を作り上げる」というプロセスを取った。それに類似した、メディア考古学と呼ばれる学問のアプローチ法がある。メディア考古学とは、考古学という名前が付いているものの特定の学会が存在するわけでは無く、アプローチ法と書いたように学問のプロセスそのものを指す言葉であり、1980年代頃からいくつかこの名前がメディア研究において使われ始めた。メディア研究者のエルキ・フータモなどが現在における代表的な研究者である。その方法自体も具体的に確立されてるわけではなく、研究者ごとに微妙に異なっているのだが、日本のメディア研究者であり、フータモの著書を訳した太田純貴は

\begin{quote}
メディア考古学を最大限一般的にしたかたちで定義すれば、「日々増殖するメディアテクノロジーによって、埋没してしまったメディア文化やそれがもたらす経験についての言説の掘り起こし」であり、「大半のメディア考古学者たちに共通するのは、メディア文化についての規範的で正統的な物語を突き抜けて「掘り下げて」、省かれたものや的外れに終わった解釈を指摘すること」とまとめることができるだろう。
\end{quote}

と述べている。\\
フータモは論文内にて積極的にメディアアートと呼ばれるモノを取り上げる{[}1{]}。メディア考古学と呼ばれる言葉の存在する以前から、メディア考古学的アプローチを取り続けているアーティストが存在すると言うのだ。\\
。\\
本論考においては「送れ|遅れ/post|past」がメディア考古学的アプローチの制作と言えるのかどうかの考察及び、「日本メディアアート史」での考察法において、メディアアートを考察する上での視点についてもメディア考古学的アプローチと言うものが考えられるということを指摘し、最終的にメディアアート史の中での本作品の位置づけというものを行う。

\section{本論考の構成}\label{ux672cux8ad6ux8003ux306eux69cbux6210}

以上の前提を元に、本論考は、以下のように構成される.

\begin{enumerate}
\def\labelenumi{\arabic{enumi}.}
\tightlist
\item
  序論(本章)
\item
  音響遅延線という装置について
\item
  卒業制作「Acoustic Delay Memory 2(仮)」の概要
\item
  制作手順と技術的解説
\item
  メディア考古学的メディアアート制作
\item
  メディア考古学的メディアアート考察
\item
  まとめ
\end{enumerate}

第2章においては、まず前提知識として卒業制作の題材となった装置である音響遅延線の技術解説と、それが作られた時代背景について解説する。\\
第3章において、卒業制作作品の全体的な解説を、3年次において制作した「Acoustic
Delay (⇔) Memory」との違いの比較を交え行う。\\
第4章においては、実際の制作の進行の記録と、作品のシステムとしての詳細な技術的解説及びそこに用いられた技術についての解説を行う。

第5章においては、「淘汰された装置を掘り起こす」という作品制作のプロセスについて、「メディア考古学的アプローチ」が作品制作とどう関わっていたかについて考察する。

第6章において、「メディア考古学的アプローチ」がメディアアートの歴史、文脈考察においてどのように関わるかを考察し、本作品がメディアアート史の中でどのような位置づけになるかについて考察したい。

第7章にて、メディアアート制作とメディアアート史を考える場合において「メディア考古学的アプローチ」がどのように適用できるのかを改めてまとめ、結びとする。

\chapter{音響遅延線という装置について}\label{ux97f3ux97ffux9045ux5ef6ux7ddaux3068ux3044ux3046ux88c5ux7f6eux306bux3064ux3044ux3066}

\section{概要}\label{ux6982ux8981}

音響遅延線メモリー(Acoustic Delay line
memory)という装置は1940\textasciitilde{}1960年頃の、初期電子計算機(いわゆる「コンピューター」)に使われていた記憶装置である。\\
(定義によって諸説あるのだが、一般的には)世界初のコンピュータと言われているENIACの次の世代として作られたアメリカのEDVAC(1951年)やイギリスのEDSAC(1949年)(EDSACはElectronic
Delay Storage Automatic
Calculatorの略であり、名前にも現れている)、また日本で最初の商用コンピュータであるFUJICでも\footnote{\section{技術解説}\label{ux6280ux8853ux89e3ux8aac}}使用されている。

開発者はENIACの開発にも携わっていたJohn Presper
Eckertであるが、はじめに導入予定であったEDVACでは特許関係で開発が遅れ、それより先に公開されていたEDVAC開発のための指針に影響を受けたEDSACが先に音響遅延線を採用したコンピュータとなった。

(適宜参考文献あげます)

音響遅延線メモリーは、水銀で満たしたタンクと、その一端に取り付けられたピエゾ素子のマイクロフォン、反対の端に取り付けられたやはりピエゾ素子のトランスデューサー(スピーカー)、及び増幅(波形整形)回路、入出力の制御回路から成り立っている。

時間を一定間隔で区切り、その間に超音波のパルスを発する/発しないを保存したい二進数データの1/0に対応させ、スピーカーから出力する。超音波のパルスは水銀中の音速、1451.4
m/sで伝わりマイクまで到達する。マイクで検出した信号を、回路の中で、パルスが存在すれば1、存在しなければ0という処理を施すと元送信した二進数データが復元される。それをもう一度同じスピーカーに送信すると、同じデータ列が循環し続けることとなる。最大で保存できるデータは区切る時間の間隔(データの送信レート)、マイクとスピーカー間の距離、音速を決定する温度に依存する。

音波の媒体として水銀を使っているのは送信、受信に使われるピエゾ素子との音響インピーダンスが近い事によって、マイクでの受信時に反射波が生じにくく、データにエラーが少ない、エネルギー効率がよい事が理由である。

記憶装置としての特徴は、逐次読み出し/書き出しであること、すなわち保存されているデータは任意のタイミングで読み出せるのではなく、一周循環してマイクの点に来るまで読み出せないことがある。\\
また、データの保存容量を増やすためにはパルスの間隔を短くすること、マイクとスピーカー間の距離を長くする、タンクを増やすのいずれかになる。そして読み出し、書き出しの速度もパルス間隔に依存し、計算自体に使う電気的な0/1の波形の速度よりは大幅に遅くならざるを得ず、仕組み上性能向上に限界が存在していた。\\
また、温度によって音速が変化してしまい、データの送信レートが変化してしまうので温度を一定に保つ仕組み(恒温槽)が必要になり、物理的に装置が巨大化せざるを得ないという欠点もあった\footnote{ただし日本のFUJICでは独自に恒温槽をなくし、読み出し速度を温度に依存して変化させることで逆転的に対応しシステムの簡略化を図っていた。}。

図とか入れる、適宜参考文献

\section{誕生から淘汰されるまでの時代の技術的移り変わり}\label{ux8a95ux751fux304bux3089ux6dd8ux6c70ux3055ux308cux308bux307eux3067ux306eux6642ux4ee3ux306eux6280ux8853ux7684ux79fbux308aux5909ux308fux308a}

EDVACに先行するENIACでは、回路を手でパッチを組みつなげることによりプログラムを構成していた。その為計算する問題を変える際は毎度パッチを組み替えないといけないなどの問題があった。

当時は真空管が不安定ですぐ焼ききれる→真空管でメモリーを組むより安価で安定\\
→磁気コアメモリの登場により台頭される、トランジスタの登場で完全に消滅

\chapter{作品概要}\label{ux4f5cux54c1ux6982ux8981}

\section{前作品「Acoustic Delay (⇔)
Memory」について}\label{ux524dux4f5cux54c1acoustic-delay-memoryux306bux3064ux3044ux3066}

本作品は前年2015年に制作した「Acoustic Delay (⇔)
Memory」に続く要素が多く含まれるので、本作品の説明に入る前に前作品について説明する。

「Acoustic Delay (⇔)
Memory」は標準的なスピーカーとマイクロフォン、簡単な電子回路及びコンピュータを使用して構成された音響装置作品である。2015年12月12、13日、音楽環境創造科の制作・研究発表会、千住
Art Path 2015 において、東京芸術大学千住キャンパスの倉庫2で展示された。

回路はトランジスタでの標準的なロジックICのみを使用し1950年代に使用された音響遅延線とほぼ等価な回路を構成している。\\
それによりマイクとスピーカーの間の遅延時間で8bit(実際には、通信のためのスタートビットとストップビットをそのまま入れているため10bit)のデータを保存し続けている。\\
PCとは標準的なRS232の一般的なシリアル通信で読み出しと書き込み(上書き)ができるようになっている。展示中はWeb上にインターフェースが用意してあり、保存されているデータ列8bitのAsciiコードに対応した文字が表示される。右側には8つの長方形が並び、ビットの1/0が白・黒の色に対応して表示されている。\\
直接その部分を編集しキーボードで文字を打ち込み、Enterキーを押すか長方形をクリックするかでデータを書き込む事もできる。\\
これは直接USBで接続されているPCからだけでなく、同じURLを開けば携帯電話からでも世界中どのコンピュータからでもデータの読み/書きができる。\\
すなわちこの音響遅延線は一種のクラウドストレージのような機能を持つ。

また、もちろん保存されているデータは音として循環し続けているので、マイクとスピーカーの間に立って経路を遮ったり、手を叩くなど大きなノイズが発されればデータは変化してしまう。

(画像) 付録にキャプションのpdfとか付けます

\section{送れ\textbar{}遅れ/post\textbar{}pastの解説}\label{ux9001ux308cux9045ux308cpostpastux306eux89e3ux8aac}

本作品の展示は2016年11月4、5、7、8、9日の5日間、東京芸術大学千住キャンパスの第7ホールにて行われた。

本作品は図に示すとおりの5台のコンピュータにより構成されるが、その役割は2台、2台、1台に分けられる。しかし個々のコンピュータは独立したプログラムによって制御され、LAN等のネットワークを介してお互いのコンピュータを制御することはない。

まず、PC1と2は2台で1つの音響遅延線としての機能を持つ。

PC3は、1の発する信号を読み取り、データに変化があったのを検知するとそれを2文字のUTF-16にエンコードし画面上に表示する。またそれがMacの標準読み上げ機能で読み上げ可能だった場合読み上げる\footnote{実のところは、展示期間5日間中で文字を表示するようになったのは2日目から、文字が正しくデコードを始めたのは3日目午後からであった。それでもPC1で書き込んだデータをそのまま表示することは一度もなかった。これについては技術的解説において考察する。}。

PC4と5は

\section{比較と考察}\label{ux6bd4ux8f03ux3068ux8003ux5bdf}

\chapter{制作手順と技術的解説}\label{ux5236ux4f5cux624bux9806ux3068ux6280ux8853ux7684ux89e3ux8aac}

\section{実際の制作スケジュール}\label{ux5b9fux969bux306eux5236ux4f5cux30b9ux30b1ux30b8ux30e5ux30fcux30eb}

7月:使える技術についての検討

\section{制作に用いられた技術}\label{ux5236ux4f5cux306bux7528ux3044ux3089ux308cux305fux6280ux8853}

QPSK(位相変異変調)のこと

(FAUSTとMaxmspでの実装のこと?)

PuredataとAppleScriptでの読み上げ機能\\
スクリーンショット

\subsection{使える技術と使えない技術}\label{ux4f7fux3048ux308bux6280ux8853ux3068ux4f7fux3048ux306aux3044ux6280ux8853}

適応フィルタに於けるマルチパス除去などをしようとすると遅延が生じることなど

\chapter{メディア考古学的メディアアート制作}\label{ux30e1ux30c7ux30a3ux30a2ux8003ux53e4ux5b66ux7684ux30e1ux30c7ux30a3ux30a2ux30a2ux30fcux30c8ux5236ux4f5c}

\section{メディア考古学とは}\label{ux30e1ux30c7ux30a3ux30a2ux8003ux53e4ux5b66ux3068ux306f}

序論でも述べたように、メディア考古学とは特定の学問領域を指すのではなく学問のアプローチである。

〜つづく〜

\section{デジタル記憶装置はメディア考古学で掘れるのか}\label{ux30c7ux30b8ux30bfux30ebux8a18ux61b6ux88c5ux7f6eux306fux30e1ux30c7ux30a3ux30a2ux8003ux53e4ux5b66ux3067ux6398ux308cux308bux306eux304b}

しかし、根本的にメディア考古学的なアプローチと言われるものと今回の作品の制作プロセスでは大きく違う部分が存在する。\\
そもそもは取り上げる装置の違いに起因する。音響遅延線だけではなく「デジタル記憶装置」という分類のものには「文化的な言及や現象」が存在していない事が多い。

その理由は主に2つである。

\begin{enumerate}
\def\labelenumi{\arabic{enumi}.}
\tightlist
\item
  記憶装置とは感性に対して直接作用するメディアでないこと。
\item
  デジタル記憶装置は扱う内容が極めて抽象化(エンコード)されること。
\end{enumerate}

「メディア考古学 ―過去と未来の対話のために―」の中で取り上げられる装置の実例を挙げると、万華鏡、ピンボールマシン、ゾートロープ(岩井俊雄の作品)、フォノグラフ(ポール・デマリニスの作品)などが挙げられる。\\
ピンボールマシンはともかく、それ以外の三者はすべて視覚や聴覚など、五感に対して何らかの作用を持つ装置であることは確かである。\\
しかし、デジタル記憶装置というものはデータをすべて一度0/1の並びに変換してその並びを保存する。\\
つまり保存するデータは言葉でも良いし、音楽でも画像でも、なんでもよい。\\
そのため、ハードディスクやフラッシュメモリ、そしてもちろん音響遅延線もレコードであったり写真のような文化を構成する一部としては考察し難い。

例えば最初は記録メディアがカセットテープであったウォークマンのことを考えてみると、1999年にフラッシュメモリ式のネットワークウォークマンと呼ばれるシリーズが発売され始める。\\
初期ウォークマンをカセットテープに付随する文化史の中に並べることはできるかもしれないが、フラッシュメモリにおいても同じ事が言えるのだろうか?\\
私自身はこれは不可能だと考える。その理由は先ほどから述べているように単にフラッシュメモリという記録メディアの保存できるものの種類が広範であることが大きい。\\
フラッシュメモリ型ウォークマンの与えた文化的影響についての考察はできるかもしれないが、フラッシュメモリ単体でそれが与えた文化的影響を考察するのは散漫にならざるをえない。\\
このように、デジタルメディアはその汎用性が故に記録、再生装置、もしくは記録/再生する内容とセットにしなければ考察する対象にならないのである。

〜つづく〜\\
結局、「送れ|遅れ/post|past」において、「手段は問わないデジタル装置」は逆説的に「何らかの伝達と認識の手段の存在」によってその機能が保証されていて、「その装置の機能の持つ手段から文化的な意味を勝手に読み出してより見える形にする」ということをしている、そしてその文化的意味が表出するためには「装置が現在の文化においては役に立たない」ことが必要となる

ということを丁寧に書きたい

\section{聴覚、音響再生産文化から見る音響遅延線}\label{ux8074ux899aux97f3ux97ffux518dux751fux7523ux6587ux5316ux304bux3089ux898bux308bux97f3ux97ffux9045ux5ef6ux7dda}

「聞こえくる過去」からのいろいろ、、、

レコードに入れた音楽は缶詰に入った食品だという比喩の話(長期保存ができるけど、生物ほどは質が良くない)的な表現との関わりとか。。。

\chapter{メディア考古学的メディアアート考察}\label{ux30e1ux30c7ux30a3ux30a2ux8003ux53e4ux5b66ux7684ux30e1ux30c7ux30a3ux30a2ux30a2ux30fcux30c8ux8003ux5bdf}

\section{メディアアートと呼ばれるモノとの交わり}\label{ux30e1ux30c7ux30a3ux30a2ux30a2ux30fcux30c8ux3068ux547cux3070ux308cux308bux30e2ux30ceux3068ux306eux4ea4ux308fux308a}

ここからはメディアアートと呼ばれるものとメディア考古学、そして本作品の関係についての考察を行いたい。\\
「メディアアートと呼ばれるもの」と書き続けているのは、「メディアアート」という言葉の定義が非常に曖昧であるからである。\\
現状メディアアートと言っても元来の意味であった「ニューメディアアート」の文脈を引き継ぐ、いわゆる新しい技術(例えば2016年に於いてはバーチャル・リアリティやディープラーニングなどはそれらに当てはまるだろう)をアート作品に用いるものや、日本の代表的なメディアアートの美術館であるNTTインターコミュニケーションセンターでの年間展示「オープン・スペース」の2016年のタイトルが「メディア・コンシャス」であるように表現に関わるテクノロジーや素材(メディウム)が現代においてどういう文化的意味を持ちうるかについて言及する傾向のものなど、幾つかの分類が可能なようにも見える。

しかしながら本論考では敢えてメディアアートの作品群を分類し、その中のどこに本作品が位置するか、のような言及の仕方は避けたいと考える。\\
なぜならそれら複数のジャンルに跨るものも当然存在するし、そのような線引きを行うことで見えなくなってしまう要素も存在すると考えるからである。(あいまいなので変えたい)

日本のメディアアート観に対する統一見解も未だまとまったものが存在するところではないが、馬定延「日本メディアアート史」などで幾つかまとめようとする試みは出てきている。\\
しかし「日本メディアアート史」でのアプローチは、一般的な美術史の様に代表的な作品を列挙するのではなく、背景的に起こった出来事(例えば大阪万博など)、セゾンや草月アートセンター、NTTのICCなど舞台となった場を中心として、メディアアートとは何なのかを探るようなアプローチとなっている。\\
このようなアプローチを取った理由について馬は序章においては

\begin{quote}
1970年代以降の日本のエレクトロニックアート紹介において、美術館における展覧会でなく、博覧会の名前が羅列されているのはなぜだろうか。なぜこれらの場を並べずには、日本のメディアアートの軌跡を語る事ができなかったのだろうか。
\end{quote}

\begin{quote}
こうした問題意識から、本書は個別の作家や作品ではなく、その背景をなす時代像に焦点を当ててみる。すなわち、本書はメディアアートの作品論と作家論を可能な限り排除して書かれたメディアアート史である。このような方法論が、究極的には、すべての表層的な要素、移り変わっていく背景を取り除いたあとに残る、メディアアートの本質たるものを強調することを目的にしていることは言うまでもない。
\end{quote}

と述べているし、またあとがきにおいては本書中で挙げられた作品写真は本文で言及されていないものばかりかつ、作品の説明などはなく最低限の情報しか載せていないことに触れた上でこう述べる。

\begin{quote}
筆者の明確な意図は、これらの画像を通して、何がメディアアートなのかではなく、むしろその定義不可能性を提示し、読者に問いかけることである。
\end{quote}

と、ある種明確にメディアアートの定義は不可能だと言い切っている。\\
実は本書は日本を代表するメディアアーティストである藤幡正樹についての研究から始まり、彼を中心として起きた出来事の歴史をまとめた論文が元になっていると本人も語っている\footnote{千住
  Art Path
  2015での毛利嘉孝とのトークにおいて。またあとがきの最後でも遠回しであるが日本で研究する目的が藤幡について研究したかったから、ということを述べている。}。その為この本もかなり引いた目線で語っているとは言え日本のメディアアートの全てを包括して語っているとは言い難い。

しかしながら私はこの「引いた目線」、歴史を一つの観点で見定めようとせずに、起きた事象を冷静に見つめ、通底する物を引き出そうとする姿勢はメディア考古学的アプローチと強く結びつくところがあると考える。その為本作品のメディアアートの中での位置づけを考える際の大きな参考の一つとしたい。

何が「引いた目線」がメディア考古学的視点と結びつくとか、と言うのは特にフータモのメディア考古学観、特に彼の言う拡張された「トポス」概念である。

〜トポス概念について書きます〜

トポス概念とはもともと文学の中の常套句のようなものであり、メディア考古学を考察するにあたってフータモはこの概念を拡張している。\\
例えばバーチャル・リアリティにおける「没入感」などの言葉や「複雑な仕組みの機械の様子を表現するのに妖精や小人がマシンを動かしている比喩」などがある

「ここ数年メディアアートは停滞している」的な言葉も散々言われ続けてるし、そういうものなのでは、、、的なことも言いたい

\section{メディアアート史における本作品の立ち位置}\label{ux30e1ux30c7ux30a3ux30a2ux30a2ux30fcux30c8ux53f2ux306bux304aux3051ux308bux672cux4f5cux54c1ux306eux7acbux3061ux4f4dux7f6e}

\chapter{まとめ}\label{ux307eux3068ux3081}

\section{結び}\label{ux7d50ux3073}

\chapter*{参考文献}\label{ux53c2ux8003ux6587ux732e}
\addcontentsline{toc}{chapter}{参考文献}

\hypertarget{refs}{}
\hypertarget{ref-huhtamo:mediaarcheology}{}
{[}1{]} E. Huhtamo, \emph{メディア考古学
過去・現在・未来の対話のために}. NTT出版, 2015.
