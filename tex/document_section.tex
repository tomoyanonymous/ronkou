\section{はじめに}\label{ux306fux3058ux3081ux306b}

私は卒業制作として、前年度から引き続き音響遅延線というコンピューターの初期に使われた記憶装置を題材にした作品制作を行った。\\
音響遅延線という装置の概要については本論で述べることにするが、一連の制作においてはすでに淘汰され使われなくなったメディア装置、その中でも特に記憶装置を取り上げ、現代の技術を取り込み別のメディア装置として再生させるという作品制作のプロセスを取った。

\subsection{本論考の構成}\label{ux672cux8ad6ux8003ux306eux69cbux6210}

本論考は、以下のように構成される。

\begin{enumerate}
\def\labelenumi{\arabic{enumi}.}
\tightlist
\item
  序論(本章)
\item
  作品の概要
\item
  音響遅延線という装置について
\item
  制作手順と技術的解説
\item
  制作のプロセス メディア考古学との交わり
\item
  無駄なモノづくりという行為への一考察
\item
  まとめ
\item
  参考文献
\end{enumerate}

本章においては制作の動機と、5、6への導入のための文章を記す。

\subsection{動機と問題意識}\label{ux52d5ux6a5fux3068ux554fux984cux610fux8b58}

今日、デジタルネイティブという言葉が存在するように私達が何か技術を使うときにアナログ、デジタルと言う境目を意識することは少なくなったように思える。また、それに伴い「保存する」ことと「通信する」事の境界は曖昧になりつつ有る。

例えば、Googleドキュメントでは書き込んだデータはリアルタイムでサーバに送信され自動で保存され、さらに共同編集をしている場合自分以外の書き込みも逐次反映されていく。\\
ここで、手紙の紙=保存メディアの発展形であったはずのGoogleドキュメントはもはや手紙という行為そのもの=通信にもなっている。そしてこれらを使うとき、石版や紙のような媒体の物質性は意識されず、人間同士のやり取りだけが意識される。\\
このような時代においてそもそも情報を保存する、伝えるとは一体どういうことなのだろうか?

情報を保存するというのは、その前提に伝えるということがあって始まるのだが、通信技術の発達と計算機の性能向上によって、わざわざ一度保存せずとも、速く遠くに伝えることができるようになってしまった。それでは何を目的として何の情報を保存する必要があるのだろうか?

その疑問に迫るために私はコンピューター最初期に使われた「通信し続けることで保存する記憶装置」をである音響遅延線を復元させるという行為を行った。

\subsection{制作のプロセスについて}\label{ux5236ux4f5cux306eux30d7ux30edux30bbux30b9ux306bux3064ux3044ux3066}

さて、今回の制作で行った「すでに淘汰されたメディア装置を研究し、新しいメディア装置の手がかりとする」というプロセスには類似したものがあり、メディア考古学と呼ばれる学問のアプローチ法がある。メディア考古学とは、考古学という名前が付いているものの特定の学会が存在するわけでは無く、アプローチ法と書いたように学問のプロセスそのものを指す言葉であり、メディア研究者のエルキ・フータモらが精力的に活動している。まだ()年に提唱されたばかりの比較的新しい概念である。\\
フータモは論文内にて積極的にメディアアートと呼ばれるモノを取り上げる。\\
本論考第5章にてこのアプローチとの共通点、相違点を考察し作品の持ち得る意味の手がかりとしたい。

\subsection{無駄であることについて}\label{ux7121ux99c4ux3067ux3042ux308bux3053ux3068ux306bux3064ux3044ux3066}

前項とも関連することだが、この作品は実際の記憶装置としての機能する前提で作られているが、その記憶容量は装置としての大きさに対して数十バイトとテキストファイルすら満足に保存することが出来ない。つまり装置の存在としてほとんど無駄と言っていい。
