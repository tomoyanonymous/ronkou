\chapter{序論}\label{ux5e8fux8ad6}

私は卒業制作として、音響遅延線メモリーというコンピューターの初期に使われていた記憶装置を題材にしたサウンドインスタレーション作品の制作を行った。\\
音響遅延線という装置の概要については本論で述べることにするが、一連の制作においてはすでに淘汰され使われなくなったメディア装置、その中でも特に記憶装置を取り上げ、現代の技術を取り込み別のメディア装置として再生させるという作品制作のプロセスを取った。

\section{本論考の構成}\label{ux672cux8ad6ux8003ux306eux69cbux6210}

本論考は、以下のように構成される.

\begin{enumerate}
\def\labelenumi{\arabic{enumi}.}
\tightlist
\item
  序論(本章)
\item
  卒業制作「Acoustic Delay Memory 2(仮)」の概要
\item
  音響遅延線という装置について
\item
  制作手順と技術的解説
\item
  制作のプロセス メディア考古学との交わり
\item
  無駄なモノづくりという行為への一考察
\item
  まとめ
\item
  参考文献
\end{enumerate}

第2章において、卒業制作作品の全体的な解説を、3年次において制作した「Acoustic
Delay (⇔) Memory」との違いの比較を交え行う。\\
第3章においては、卒業制作の題材となった装置である音響遅延線の技術解説と、それが作られた時代背景について解説する。\\
第4章においては、作品のシステムとしての詳細な技術的解説及びそこに用いられた技術についての解説を行う。

第5章においては、「淘汰された装置を掘り起こす」という作品制作のプロセスについて、メディア考古学という学問のアプローチをヒントに作品制作の思想について、またメディアアートと呼ばれる作品群の中での本作品の立ち位置について参考作品との比較も交え論じる。

また第6章において本作品が学術研究として根本的に無駄であることについて、近年のメイカーズムーブメントと呼ばれる現象やニコニコ学会βなどに代表される「野生の研究者」という概念との関わりを比較しつつ、作品制作の意義について論じる

第7章にて。

本章以下においては制作の動機と、第5、6章への導入のための文章を記す。

\section{動機と問題意識}\label{ux52d5ux6a5fux3068ux554fux984cux610fux8b58}

今日、デジタルネイティブという言葉が存在するように私達が何か技術を使うときにアナログ、デジタルと言う境目を意識することは少なくなったように思える。また、それに伴い「保存する」ことと「通信する」事の境界は曖昧になりつつ有る。

例えば、Googleドキュメントでは書き込んだデータはリアルタイムでサーバに送信され自動で保存され、さらに共同編集をしている場合自分以外の書き込みも逐次反映されていく。\\
ここで、手紙の紙=保存メディアの発展形であったはずのGoogleドキュメントはもはや手紙という行為そのもの=通信にもなっている。そしてこれらを使うとき、石版や紙のような媒体の物質性は意識されず、人間同士のやり取りだけが意識される。\\
このような時代においてそもそも情報を保存する、伝えるとは一体どういうことなのだろうか?

情報を保存するというのは、その前提に伝えるということがあって始まるのだが、通信技術の発達と計算機の性能向上によって、わざわざ一度保存せずとも、速く遠くに伝えることができるようになってしまった。それでは何を目的として何の情報を保存する必要があるのだろうか?

その疑問に迫るために私はコンピューター最初期に使われた「通信し続けることで保存する記憶装置」をである音響遅延線を復元させるという行為を行った。

\section{前年度までの取り組み}\label{ux524dux5e74ux5ea6ux307eux3067ux306eux53d6ux308aux7d44ux307f}

私は2015年度にサウンドインスタレーション作品「Acoustic Delay (⇔)
Memory」を制作した。本卒業制作のはこの作品をベースに更に発展したもので、1950年代に使われた音響遅延線という装置を

\section{制作のプロセスについて}\label{ux5236ux4f5cux306eux30d7ux30edux30bbux30b9ux306bux3064ux3044ux3066}

さて、今回の制作で行った「すでに淘汰されたメディア装置を研究し、新しいメディア装置の手がかりとする」というプロセスには類似したものがあり、メディア考古学と呼ばれる学問のアプローチ法がある。メディア考古学とは、考古学という名前が付いているものの特定の学会が存在するわけでは無く、アプローチ法と書いたように学問のプロセスそのものを指す言葉であり、メディア研究者のエルキ・フータモらが精力的に活動している。まだ()年に提唱されたばかりの比較的新しい概念である。\\
フータモは論文内にて積極的にメディアアートと呼ばれるモノを取り上げる。\\
本論考第5章にてこのアプローチとの共通点、相違点を考察し作品の持ち得る意味の手がかりとしたい。

\section{無駄であることについて}\label{ux7121ux99c4ux3067ux3042ux308bux3053ux3068ux306bux3064ux3044ux3066}

前項とも関連することだが、この作品は実際の記憶装置としての機能する前提で作られているが、その記憶容量は装置としての大きさに対して数十バイトとテキストファイルすら満足に保存することが出来ない。つまり装置の存在としてほとんど無駄と言っていい。

\chapter{作品解説}\label{ux4f5cux54c1ux89e3ux8aac}

\section{概要}\label{ux6982ux8981}

\section{3年次作品「Acoustic Delay (⇔)
Memory」について}\label{ux5e74ux6b21ux4f5cux54c1acoustic-delay-memoryux306bux3064ux3044ux3066}

\section{比較と考察}\label{ux6bd4ux8f03ux3068ux8003ux5bdf}

\chapter{音響遅延線という装置について}\label{ux97f3ux97ffux9045ux5ef6ux7ddaux3068ux3044ux3046ux88c5ux7f6eux306bux3064ux3044ux3066}

\section{技術的概要}\label{ux6280ux8853ux7684ux6982ux8981}

\section{誕生から淘汰されるまでの時代の技術的移り変わり}\label{ux8a95ux751fux304bux3089ux6dd8ux6c70ux3055ux308cux308bux307eux3067ux306eux6642ux4ee3ux306eux6280ux8853ux7684ux79fbux308aux5909ux308fux308a}

\chapter{制作手順と技術的解説}\label{ux5236ux4f5cux624bux9806ux3068ux6280ux8853ux7684ux89e3ux8aac}

\section{実際の制作スケジュール}\label{ux5b9fux969bux306eux5236ux4f5cux30b9ux30b1ux30b8ux30e5ux30fcux30eb}

\section{制作の基本的ルーチン―これを研究と呼ぶ事ができるか?}\label{ux5236ux4f5cux306eux57faux672cux7684ux30ebux30fcux30c1ux30f3ux3053ux308cux3092ux7814ux7a76ux3068ux547cux3076ux4e8bux304cux3067ux304dux308bux304b}

\chapter{制作のプロセス―メディア考古学との交わり}\label{ux5236ux4f5cux306eux30d7ux30edux30bbux30b9ux30e1ux30c7ux30a3ux30a2ux8003ux53e4ux5b66ux3068ux306eux4ea4ux308fux308a}

\section{メディア考古学とは}\label{ux30e1ux30c7ux30a3ux30a2ux8003ux53e4ux5b66ux3068ux306f}

\section{メディアアートと呼ばれるモノとの交わり}\label{ux30e1ux30c7ux30a3ux30a2ux30a2ux30fcux30c8ux3068ux547cux3070ux308cux308bux30e2ux30ceux3068ux306eux4ea4ux308fux308a}

\section{メディアアートにおける本作品の立ち位置}\label{ux30e1ux30c7ux30a3ux30a2ux30a2ux30fcux30c8ux306bux304aux3051ux308bux672cux4f5cux54c1ux306eux7acbux3061ux4f4dux7f6e}

\chapter{無駄なモノづくりという行為への一考察}\label{ux7121ux99c4ux306aux30e2ux30ceux3065ux304fux308aux3068ux3044ux3046ux884cux70baux3078ux306eux4e00ux8003ux5bdf}

\section{メイカーズムーブメントについて}\label{ux30e1ux30a4ux30abux30fcux30baux30e0ux30fcux30d6ux30e1ux30f3ux30c8ux306bux3064ux3044ux3066}

\section{「野生の研究者」という概念}\label{ux91ceux751fux306eux7814ux7a76ux8005ux3068ux3044ux3046ux6982ux5ff5}

\section{無駄な研究とアート制作の関係についての考察}\label{ux7121ux99c4ux306aux7814ux7a76ux3068ux30a2ux30fcux30c8ux5236ux4f5cux306eux95a2ux4fc2ux306bux3064ux3044ux3066ux306eux8003ux5bdf}

\chapter{まとめ}\label{ux307eux3068ux3081}

\chapter{参考文献}\label{ux53c2ux8003ux6587ux732e}
