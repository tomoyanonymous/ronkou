\section{論文概要}\label{ux8ad6ux6587ux6982ux8981}

メディア考古学的アプローチからのメディアアート制作

\section{はじめに}\label{ux306fux3058ux3081ux306b}

私は卒業制作として、前年度から引き続き音響遅延線というコンピューターの初期に使われた記憶装置を題材にした作品制作を行った。\\
音響遅延線という装置の概要については本論で述べることにするが、一連の制作においてはすでに淘汰され使われなくなったメディア装置、その中でも特に記憶装置を取り上げ、現代の技術を取り込み別のメディア装置として再生させるという作品制作のプロセスを取った。

これに類似したプロセスとして、メディア考古学と呼ばれる学問のアプローチ法がある。メディア考古学とは、考古学という名前が付いているものの特定の学会が存在するわけでは無く、アプローチ法と書いたように学問のプロセスそのものを指す言葉であり、メディア研究者のエルキ・フータモらが精力的に活動している。まだ()年に提唱されたばかりの比較的新しい概念である。

メディア考古学の特徴としてメディア・アートと呼ばれるものを積極的に取り上げるということがある。フータモはメディア考古学は言葉としては新しいものであるが、メディア考古学的アプローチを無意識に取り続けているアーティストはメディア考古学という言葉が誕生する以前から存在すると述べている。彼の論文をまとめた著書「メディア考古学」では岩井俊雄{[}\^{}note1{]}やPaul
Demanirisらが挙げられている。

\begin{center}\rule{0.5\linewidth}{\linethickness}\end{center}

\subsection{デジタルにおける、保存と通信}\label{ux30c7ux30b8ux30bfux30ebux306bux304aux3051ux308bux4fddux5b58ux3068ux901aux4fe1}

今日、デジタルネイティブという言葉が存在するように私達が何か技術を使うときにアナログ、デジタルと言う境目を意識することは少なくなったように思える。また、それに伴い「保存する」ことと「通信する」事の境界は曖昧になりつつ有る。\\
例えば、Googleドキュメントでは書き込んだデータはリアルタイムでサーバに送信され自動で保存され、さらに共同編集をしている場合自分以外の書き込みも逐次反映されていく。\\
ここで、手紙の紙=保存メディアの発展形であったはずのGoogleドキュメントはもはや手紙という行為そのもの=通信にもなっている。そしてこれらを使うとき、石版や紙のような媒体の物質性は意識されず、人間同士のやり取りだけが意識される。

\subsection{デジタルにおける、物質性と非物質性}\label{ux30c7ux30b8ux30bfux30ebux306bux304aux3051ux308bux7269ux8ceaux6027ux3068ux975eux7269ux8ceaux6027}

デジタルとは本来扱う量が離散的であること、digitから来ているが、現代の用法ではデジタル計算機であるコンピュータやマイコンなどが使われていればデジタルと呼ばれるようになりつつあり、文化的に意味が拡散しつつ有る。\\
研究者かつメディアアーティストの落合陽一は「デジタルネイチャー」という(計算機の高性能化に伴い新たな自然現象のように見えるとかそういう)概念を提唱しているが、本人によると「コンピューテーショナルネイチャー」のほうが良いんだけど長いからという理由でデジタルネイチャーと呼んでいると語る。また、本来の意味での「digit」という意味でデジタルとは言っていないと述べている。(http://www.todaishimbun.org/ochiai151105/)
このように「デジタル」の持つ意味は一般からも専門家からも拡散しつつあると言える。

\subsection{デジタルにおける、身体性と非身体性}\label{ux30c7ux30b8ux30bfux30ebux306bux304aux3051ux308bux8eabux4f53ux6027ux3068ux975eux8eabux4f53ux6027}

コンピュータ時代のユーザーインターフェース研究者である渡邊恵太は「融けるデザイン」において、

このような時代においてそもそも情報を保存する、伝えるとは一体何なのかについて考えたい。その上で私は逆説的にプリミティブなデジタルメディアを取り上げたい。

\subsection{無駄であること}\label{ux7121ux99c4ux3067ux3042ux308bux3053ux3068}

\subsection{論文の構成}\label{ux8ad6ux6587ux306eux69cbux6210}

本論文は、以降は以下のように構成される。

\begin{enumerate}
\def\labelenumi{\arabic{enumi}.}
\tightlist
\item
  序論
\item
  制作手順
\item
  考察
\item
  結論
\item
  参考文献
\end{enumerate}

\section{考察}\label{ux8003ux5bdf}

\section{結論}\label{ux7d50ux8ad6}

\section{参考文献}\label{ux53c2ux8003ux6587ux732e}
