\chapter{序論}\label{ux5e8fux8ad6}

私は卒業制作として、音響遅延線メモリーというコンピューターの初期に使われていた記憶装置を題材にしたサウンドインスタレーション作品の制作を行った。\\
音響遅延線という装置の概要については本論で述べることにするが、一連の制作においてはすでに淘汰され使われなくなったメディア装置、その中でも特に記憶装置を取り上げ、現代の技術を取り込み別のメディア装置として再生させるという作品制作のプロセスを取った。

\section{本論考の構成}\label{ux672cux8ad6ux8003ux306eux69cbux6210}

本論考は、以下のように構成される.

\begin{enumerate}
\def\labelenumi{\arabic{enumi}.}
\tightlist
\item
  序論(本章)
\item
  卒業制作「Acoustic Delay Memory 2(仮)」の概要
\item
  音響遅延線という装置について
\item
  制作手順と技術的解説
\item
  制作のプロセス メディア考古学との交わり
\item
  無駄なモノづくりという行為への一考察
\item
  まとめ
\item
  参考文献
\end{enumerate}

第2章において、卒業制作作品の全体的な解説を、3年次において制作した「Acoustic
Delay (⇔) Memory」との違いの比較を交え行う。\\
第3章においては、卒業制作の題材となった装置である音響遅延線の技術解説と、それが作られた時代背景について解説する。\\
第4章においては、作品のシステムとしての詳細な技術的解説及びそこに用いられた技術についての解説を行う。

第5章においては、「淘汰された装置を掘り起こす」という作品制作のプロセスについて、メディア考古学という学問のアプローチをヒントに作品制作の思想について、またメディアアートと呼ばれる作品群の中での本作品の立ち位置について参考作品との比較も交え論じる。

また第6章において本作品が学術研究として根本的に無駄であることについて、近年のメイカーズムーブメントと呼ばれる現象やニコニコ学会βなどに代表される「野生の研究者」という概念との関わりを比較しつつ、作品制作の意義について論じる。

第7章にて、前項までの議論をまとめつつ、作品として出来上がった「モノ」は何だったのか、どういう意味を持つのかについてと、一連の制作という「行為」が何だったのかについて論じたい。

本章以下においては制作の動機と、第5、6章への導入のための文章を記す。

\section{動機と問題意識}\label{ux52d5ux6a5fux3068ux554fux984cux610fux8b58}

今日、デジタルネイティブという言葉が存在するように私達が何か技術を使うときにアナログ、デジタルと言う境目を意識することは少なくなったように思える。また、それに伴い「保存する」ことと「通信する」事の境界は曖昧になりつつ有る。

例えば、Googleドキュメントでは書き込んだデータはリアルタイムでサーバに送信され自動で保存され、さらに共同編集をしている場合自分以外の書き込みも逐次反映されていく。\\
ここで、手紙の紙=保存メディアの発展形であったはずのGoogleドキュメントはもはや手紙という行為そのもの=通信にもなっている。そしてこれらを使うとき、石版や紙のような媒体の物質性は意識されず、人間同士のやり取りだけが意識される。\\
このような時代においてそもそも情報を保存する、伝えるとは一体どういうことなのだろうか?

情報を保存するというのは、その前提に伝えるということがあって始まるのだが、通信技術の発達と計算機の性能向上によって、わざわざ一度保存せずとも、速く遠くに伝えることができるようになってしまった。それでは何を目的として何の情報を保存する必要があるのだろうか?

その疑問に迫るために私はコンピューター最初期に使われた「通信し続けることで保存する記憶装置」をである音響遅延線を復元させるという行為を行った。

\section{研究であるか、制作であるか?}\label{ux7814ux7a76ux3067ux3042ux308bux304bux5236ux4f5cux3067ux3042ux308bux304b}

この作品制作においての大きな特徴といえるのは、装置を作るのに集中している間はそれほど「アート作品をつくる」という意識を持たずに制作していたことだと考える。記憶装置を作る間は記憶装置としての性能の向上のための技術リサーチと、その実装における試行錯誤の繰り返しであった。これはどちらかと言えば理系の研究に近いルーチンではないかと考える。しかし、試行錯誤の末出来上がる装置は実用的に見れば完全に役立たずなものである(詳しい理由については本論にて述べる)。では私の行った行為は一体何だったのだろう?

\section{メディア考古学という学問が与えるヒント}\label{ux30e1ux30c7ux30a3ux30a2ux8003ux53e4ux5b66ux3068ux3044ux3046ux5b66ux554fux304cux4e0eux3048ux308bux30d2ux30f3ux30c8}

さて、この疑問に対して一つのヒントと生る学問の分野が有る。今回の制作で行った「すでに淘汰されたメディア装置を研究し、新しいメディア装置の手がかりとする」というプロセスに類似した、メディア考古学と呼ばれる学問のアプローチ法がある。メディア考古学とは、考古学という名前が付いているものの特定の学会が存在するわけでは無く、アプローチ法と書いたように学問のプロセスそのものを指す言葉であり、1980年代頃から積極的にこの名前で研究がされ始めた。メディア研究者のエルキ・フータモなどが現在における代表的な研究者である。\\
フータモは論文内にて積極的にメディアアートと呼ばれるモノを取り上げる。メディア考古学と呼ばれる言葉の存在する以前から、メディア考古学的アプローチを取り続けているアーティストが存在すると言うのだ。\\
本論考第5章にてこのアプローチとの共通点、相違点を考察し本作品がメディアアートとしてどう振る舞うのかについて論じたいと思う。

\section{無駄なモノづくり}\label{ux7121ux99c4ux306aux30e2ux30ceux3065ux304fux308a}

もう一つのヒントとして、そもそも学術的研究では無かったものを研究と近づけて捉える動きがある。代表的なものが80年代からメディアアーティストして活動する一方産業技術総合研究所の研究員として活動する江渡浩一郎の提唱する「野生の研究者」という概念だ。彼は「研究に対する姿勢」が研究者を定義づけるとしてニコニコ動画{[}\^{}note1{]}で好き勝手にモノづくりをする人たちの発表の場として「ニコニコ学会β」という団体を立ち上げ「ユーザー参加型研究」を推進してきた。この中で発表される研究には「NHKだけ映らないアンテナの開発」や「全てを雌しべと雄しべで説明するプログラミング言語を作ってみた」など、およそ人類の未来のためになるとは思えないものも含まれる。だが、すべてを体系立て、効率化するのでは無い無駄な行動に文化の発展の可能性があるのではないか?ということと、野生の研究という視点から見た今回の作品制作について第6章にて論じたい。

\chapter{作品解説}\label{ux4f5cux54c1ux89e3ux8aac}

\section{概要}\label{ux6982ux8981}

\section{3年次作品「Acoustic Delay (⇔)
Memory」について}\label{ux5e74ux6b21ux4f5cux54c1acoustic-delay-memoryux306bux3064ux3044ux3066}

\section{比較と考察}\label{ux6bd4ux8f03ux3068ux8003ux5bdf}

\chapter{音響遅延線という装置について}\label{ux97f3ux97ffux9045ux5ef6ux7ddaux3068ux3044ux3046ux88c5ux7f6eux306bux3064ux3044ux3066}

\section{技術的概要}\label{ux6280ux8853ux7684ux6982ux8981}

\section{誕生から淘汰されるまでの時代の技術的移り変わり}\label{ux8a95ux751fux304bux3089ux6dd8ux6c70ux3055ux308cux308bux307eux3067ux306eux6642ux4ee3ux306eux6280ux8853ux7684ux79fbux308aux5909ux308fux308a}

\chapter{制作手順と技術的解説}\label{ux5236ux4f5cux624bux9806ux3068ux6280ux8853ux7684ux89e3ux8aac}

\section{実際の制作スケジュール}\label{ux5b9fux969bux306eux5236ux4f5cux30b9ux30b1ux30b8ux30e5ux30fcux30eb}

\section{制作の基本的ルーチン―これを研究と呼ぶ事ができるか?}\label{ux5236ux4f5cux306eux57faux672cux7684ux30ebux30fcux30c1ux30f3ux3053ux308cux3092ux7814ux7a76ux3068ux547cux3076ux4e8bux304cux3067ux304dux308bux304b}

\chapter{制作のプロセス―メディア考古学との交わり}\label{ux5236ux4f5cux306eux30d7ux30edux30bbux30b9ux30e1ux30c7ux30a3ux30a2ux8003ux53e4ux5b66ux3068ux306eux4ea4ux308fux308a}

\section{メディア考古学とは}\label{ux30e1ux30c7ux30a3ux30a2ux8003ux53e4ux5b66ux3068ux306f}

\section{メディアアートと呼ばれるモノとの交わり}\label{ux30e1ux30c7ux30a3ux30a2ux30a2ux30fcux30c8ux3068ux547cux3070ux308cux308bux30e2ux30ceux3068ux306eux4ea4ux308fux308a}

\section{メディアアートにおける本作品の立ち位置}\label{ux30e1ux30c7ux30a3ux30a2ux30a2ux30fcux30c8ux306bux304aux3051ux308bux672cux4f5cux54c1ux306eux7acbux3061ux4f4dux7f6e}

現代、メディアアートと呼ばれる言葉は拡散し続ける傾向にあるが、2つの傾向として、本来日本でメディアアートを指していたのに近い「新しい技術」(例えば2016年現在においては、ディープラーニングやVRの技術がそうだろう)を用いる「ニューメディアアート」と、技術的に新しいかどうかよりも、表現の技法や素材(メディウム)が現代においてどういう文化的意味を持ちうるかについて言及する、より現代アートとしての傾向が強い「コンテンポラリーメディアアート」が存在すると筆者は考える(もちろんこの区分とこの言葉は筆者が勝手に定義及び再定義したものである)。

\chapter{無駄なモノづくりという行為への一考察}\label{ux7121ux99c4ux306aux30e2ux30ceux3065ux304fux308aux3068ux3044ux3046ux884cux70baux3078ux306eux4e00ux8003ux5bdf}

\section{メイカーズムーブメントについて}\label{ux30e1ux30a4ux30abux30fcux30baux30e0ux30fcux30d6ux30e1ux30f3ux30c8ux306bux3064ux3044ux3066}

\section{「野生の研究者」という概念}\label{ux91ceux751fux306eux7814ux7a76ux8005ux3068ux3044ux3046ux6982ux5ff5}

\section{無駄な研究とアート制作の関係についての考察}\label{ux7121ux99c4ux306aux7814ux7a76ux3068ux30a2ux30fcux30c8ux5236ux4f5cux306eux95a2ux4fc2ux306bux3064ux3044ux3066ux306eux8003ux5bdf}

\chapter{まとめ}\label{ux307eux3068ux3081}

\section{今回作った作品の持ちうる意味}\label{ux4ecaux56deux4f5cux3063ux305fux4f5cux54c1ux306eux6301ux3061ux3046ux308bux610fux5473}

\section{今回の作品制作の行為の持ちうる意味}\label{ux4ecaux56deux306eux4f5cux54c1ux5236ux4f5cux306eux884cux70baux306eux6301ux3061ux3046ux308bux610fux5473}

\chapter{参考文献}\label{ux53c2ux8003ux6587ux732e}
