\chapter{序論}\label{ux5e8fux8ad6}

筆者は卒業制作として、音響遅延線メモリーというコンピューターの初期に使われていた記憶装置を題材にした音響装置作品の制作を行った。\\
音響遅延線という装置の概要については本論で述べることにするが、一連の制作においてはすでに淘汰され使われなくなったメディア装置、その中でも特に記憶装置を取り上げ、現代の技術を取り込み別のメディア装置として再生させるという作品制作のプロセスを取った。

\section{動機と問題意識}\label{ux52d5ux6a5fux3068ux554fux984cux610fux8b58}

今日、デジタルネイティブという言葉が存在するように我々が何か技術を使うときにアナログ、デジタルと言う境目を意識することは少なくなったように思える。また、それに伴い「保存する」ことと「通信する」事の境界は曖昧になりつつ有る。

例えば、MicrosoftのWordに代表される文書作成ソフトをWebブラウザ上で動作させるようにしたGoogle
Docs\footnote{Googleの提供する、ウェブブラウザ上で動作する文書作成ソフト。Wordなどの既存のフォーマットを扱うこともできる。\url{https://www.google.co.jp/intl/ja/docs/about/}}では書き込んだデータはリアルタイムでサーバに送信され自動で保存され、さらに共同編集をしている場合自分以外の書き込みも逐次反映されていく。\\
ここで、手紙の紙=保存メディアの発展形であったはずのGoogle
Docsはもはや手紙という行為そのもの=通信にもなっている。そしてこれらを使うとき、石版や紙のような媒体の物質性は意識されず、人間同士のやり取りだけが意識される。\\
このような時代においてそもそも情報を保存する、伝えるとは一体どういうことなのだろうか?

情報を保存するというのは、その前提に伝えるということがあって始まるのだが、通信技術の発達と計算機の性能向上によって、わざわざ一度保存せずとも、速く遠くに伝えることができるようになってしまった。それでは何を目的として何の情報を保存する必要があるのだろうか?

その疑問に迫るために筆者はコンピューター最初期に使われた「通信し続けることで保存する記憶装置」をである音響遅延線を復元させるという行為を行った。

\section{メディア考古学という学問アプローチ}\label{ux30e1ux30c7ux30a3ux30a2ux8003ux53e4ux5b66ux3068ux3044ux3046ux5b66ux554fux30a2ux30d7ux30edux30fcux30c1}

この疑問に対して一つのヒントとなる学問の分野が有る。今回の制作では「すでに淘汰されたメディア装置を研究し、別の形のメディア装置を作り上げる」というプロセスを取った。それに類似した、メディア考古学と呼ばれる学問のアプローチ法がある。メディア考古学とは、考古学という名前が付いているものの特定の学会が存在するわけでは無く、アプローチ法と書いたように学問のプロセスそのものを指す言葉であり、1980年代頃から積極的にこの名前で研究がされ始めた。メディア研究者のエルキ・フータモなどが現在における代表的な研究者である。その方法自体も具体的に確立されてるわけではなく、研究者ごとに微妙に異なっているのだが、日本のメディア研究者であり、フータモの著書を訳した太田純貴は

\begin{quote}
メディア考古学を最大限一般的にしたかたちで定義すれば、「日々増殖するメディアテクノロジーによって、埋没してしまったメディア文化やそれがもたらす経験についての言説の掘り起こし」であり、「大半のメディア考古学者たちに共通するのは、メディア文化についての規範的で正統的な物語を突き抜けて「掘り下げて」、省かれたものや的外れに終わった解釈を指摘すること」とまとめることができるだろう。
\end{quote}

と述べている。\\
フータモは論文内にて積極的にメディアアートと呼ばれるモノを取り上げる。メディア考古学と呼ばれる言葉の存在する以前から、メディア考古学的アプローチを取り続けているアーティストが存在すると言うのだ。\\
メディアアートという語の指す範囲は(こと日本における「メディア芸術」はエンタテインメント作品やマンガ、アニメ、ゲームも含むためさらに)拡散しつつある。しかし筆者の作った作品をアートの中に含めるのであれば作品の題材に記録メディアを用いている以上メディアアートの中に含まれると言っていいだろう。その広いメディアアートの中で今回の作品はどのような位置づけになるかを考察する。\\

\section{本論考の構成}\label{ux672cux8ad6ux8003ux306eux69cbux6210}

以上の前提を元に、本論考は、以下のように構成される.

\begin{enumerate}
\def\labelenumi{\arabic{enumi}.}
\tightlist
\item
  序論(本章)
\item
  音響遅延線という装置について
\item
  卒業制作「Acoustic Delay Memory 2(仮)」の概要
\item
  制作手順と技術的解説
\item
  制作のプロセス―メディア考古学との交わり
\item
  まとめ
\end{enumerate}

第2章においては、まず前提知識として卒業制作の題材となった装置である音響遅延線の技術解説と、それが作られた時代背景について解説する。\\
第3章において、卒業制作作品の全体的な解説を、3年次において制作した「Acoustic
Delay (⇔) Memory」との違いの比較を交え行う。\\
第4章においては、実際の制作の進行の記録と、作品のシステムとしての詳細な技術的解説及びそこに用いられた技術についての解説を行う。

第5章においては、「淘汰された装置を掘り起こす」という作品制作のプロセスについて、メディア考古学という学問のアプローチをヒントに作品制作の思想について、またメディアアートと呼ばれる作品群の中で、この作品の立ち位置について参考作品との比較も交え論じる。

第6章にて、前項までの議論をまとめつつ、作品として出来上がった「モノ」は何だったのか、どういう意味を持つのかについてと、一連の制作という「行為」が何だったのかについて論じたい。そしてこれらのある意味で一歩引いたメタ的作品考察を元に、改めて制作動機で述べた疑問と作品制作の関係について論じて結びとする。

\chapter{音響遅延線という装置について}\label{ux97f3ux97ffux9045ux5ef6ux7ddaux3068ux3044ux3046ux88c5ux7f6eux306bux3064ux3044ux3066}

\section{概要}\label{ux6982ux8981}

音響遅延線メモリー(Acoustic Delay line
memory)という装置は1940\textasciitilde{}1960年頃の、初期電子計算機(いわゆる「コンピューター」)に使われていた記憶装置である。\\
(定義によって諸説あるのだが、一般的には)世界初のコンピュータと言われているENIACの次の世代として作られたアメリカのEDVAC(1951年)やイギリスのEDSAC(1949年)(EDSACはElectronic
Delay Storage Automatic
Calculatorの略であり、名前にも現れている)、また日本で最初の商用コンピュータであるFUJICでも\footnote{\section{技術解説}\label{ux6280ux8853ux89e3ux8aac}}使用されている。

開発者はENIACの開発にも携わっていたJohn Presper
Eckertであるが、はじめに導入予定であったEDVACでは特許関係で開発が遅れ、それより先に公開されていたEDVAC開発のための指針に影響を受けたEDSACが先に音響遅延線を採用したコンピュータとなった。

音響遅延線メモリーは、水銀で満たしたタンクと、その一端に取り付けられたピエゾ素子のマイクロフォン、反対の端に取り付けられたやはりピエゾ素子のトランスデューサー(スピーカー)、及び増幅(波形整形)回路、入出力の制御回路から成り立っている。

時間を一定間隔で区切り、その間に超音波のパルスを発する/発しないを保存したい二進数データの1/0に対応させ、スピーカーから出力する。超音波のパルスは水銀中の音速、1451.4
m/sで伝わりマイクまで到達する。マイクで検出した信号を、回路の中で、パルスが存在すれば1、存在しなければ0という処理を施すと元送信した二進数データが復元される。それをもう一度同じスピーカーに送信すると、同じデータ列が循環し続けることとなる。最大で保存できるデータは区切る時間の感覚(データの送信レート)、マイクとスピーカー間の距離、音速を決定する温度に依存する。

音波の媒体として水銀を使っているのは送信、受信に使われるピエゾ素子との音響インピーダンスが近い事によって、マイクでの受信時に反射波が生じにくく、データにエラーが少ない、エネルギー効率がよい事が理由である(現代で再現する実験をする際には水銀の危険性などを考慮してニッケルなどの合金を使用している)。

図とか入れる

\section{誕生から淘汰されるまでの時代の技術的移り変わり}\label{ux8a95ux751fux304bux3089ux6dd8ux6c70ux3055ux308cux308bux307eux3067ux306eux6642ux4ee3ux306eux6280ux8853ux7684ux79fbux308aux5909ux308fux308a}

EDVACに先行するENIACでは、回路を手でパッチを組みつなげることによりプログラムを構成していた。その為計算する問題を変える際は毎度パッチを組み替えないといけないなどの問題があった。\\
また

当時は真空管が不安定ですぐ焼ききれる→真空管でメモリーを組むより安価で安定\\
→磁気コアメモリの登場により台頭される、トランジスタの登場で完全に消滅

\chapter{作品概要}\label{ux4f5cux54c1ux6982ux8981}

\section{前作品「Acoustic Delay (⇔)
Memory」について}\label{ux524dux4f5cux54c1acoustic-delay-memoryux306bux3064ux3044ux3066}

本作品は前年2015年に制作した「Acoustic Delay (⇔)
Memory」に続く要素が多く含まれるので、本作品の説明に入る前に前作品について説明する。

「Acoustic Delay (⇔)
Memory」は標準的なスピーカーとマイクロフォン、簡単な電子回路及びコンピュータを使用して構成された音響装置作品である。

回路はトランジスタでの標準的なロジックICのみを使用しほぼ音響遅延線と等価な回路を構成している。\\
それによりマイクとスピーカーの間の遅延時間で8bit(実際には、通信のためのスタートビットとストップビットをそのまま入れているため10bit)のデータを保存し続けている。\\
PCとは標準的なRS232の一般的なシリアル通信で読み出しと書き込み(上書き)ができるようになっている。展示中はWeb上にインターフェースが用意してあり、保存されているデータ列8bitのAsciiコードに対応した文字が表示される。右側には8つの長方形が並び、ビットの1/0が白・黒の色に対応して表示されている。\\
直接その部分を編集しキーボードで文字を打ち込み、Enterキーを押すか長方形をクリックするかでデータを書き込む事もできる。\\
これは直接USBで接続されているPCからだけでなく、同じURLを開けば携帯電話からでも世界中どのコンピュータからでもデータの読み/書きができる。\\
すなわちこの音響遅延線は一種のクラウドストレージのような機能を持つ。

また、もちろん保存されているデータは音として循環し続けているので、マイクとスピーカーの間に立って経路を遮ったり、手を叩くなど大きなノイズが発されればデータは変化してしまう。

\section{送れ\textbar{}遅れ/post\textbar{}pastの解説}\label{ux9001ux308cux9045ux308cpostpastux306eux89e3ux8aac}

\section{比較と考察}\label{ux6bd4ux8f03ux3068ux8003ux5bdf}

\chapter{制作手順と技術的解説}\label{ux5236ux4f5cux624bux9806ux3068ux6280ux8853ux7684ux89e3ux8aac}

\section{実際の制作スケジュール}\label{ux5b9fux969bux306eux5236ux4f5cux30b9ux30b1ux30b8ux30e5ux30fcux30eb}

7月:使える技術についての検討

\section{制作の基本的ルーチン―これを研究と呼ぶ事ができるか?}\label{ux5236ux4f5cux306eux57faux672cux7684ux30ebux30fcux30c1ux30f3ux3053ux308cux3092ux7814ux7a76ux3068ux547cux3076ux4e8bux304cux3067ux304dux308bux304b}

\section{制作に用いられた技術}\label{ux5236ux4f5cux306bux7528ux3044ux3089ux308cux305fux6280ux8853}

\subsection{使える技術と使えない技術}\label{ux4f7fux3048ux308bux6280ux8853ux3068ux4f7fux3048ux306aux3044ux6280ux8853}

\chapter{制作のプロセス―メディア考古学との交わり}\label{ux5236ux4f5cux306eux30d7ux30edux30bbux30b9ux30e1ux30c7ux30a3ux30a2ux8003ux53e4ux5b66ux3068ux306eux4ea4ux308fux308a}

\section{メディア考古学とは}\label{ux30e1ux30c7ux30a3ux30a2ux8003ux53e4ux5b66ux3068ux306f}

序論でも述べたように、メディア考古学とは特定の学問領域を指すのではなく学問のアプローチである。

\section{デジタル記憶装置をメディア考古学で掘れるのか}\label{ux30c7ux30b8ux30bfux30ebux8a18ux61b6ux88c5ux7f6eux3092ux30e1ux30c7ux30a3ux30a2ux8003ux53e4ux5b66ux3067ux6398ux308cux308bux306eux304b}

しかし、根本的にメディア考古学的なアプローチと言われるものと今回の作品の制作プロセスでは大きく違う部分が存在する。\\
そもそもは取り上げる装置の違いに起因する。音響遅延線だけではなく「デジタル記憶装置」という分類のものには「文化的な言及や現象」が存在していない事が多い。

その理由は主に2つである。

\begin{enumerate}
\def\labelenumi{\arabic{enumi}.}
\tightlist
\item
  記憶装置とは感性に対して直接作用するメディアでないこと。
\item
  デジタル記憶装置は扱う内容が極めて抽象化(エンコード)されること。
\end{enumerate}

「メディア考古学 ―過去と未来の対話のために―」の中で取り上げられる装置の実例を挙げると、万華鏡、ピンボールマシン、ゾートロープ(岩井俊雄の作品)、フォノグラフ(ポール・デマリニスの作品)などが挙げられる。\\
ピンボールマシンはともかく、それ以外の三者はすべて視覚や聴覚など、五感に対して何らかの作用を持つ装置であることは確かである。\\
しかし、デジタル記憶装置というものはデータをすべて一度0/1の並びに変換してその並びを保存する。\\
つまり保存するデータは言葉でも良いし、音楽でも画像でも、なんでもよい。\\
このプロセスは翻訳的とも言うことができ、感覚というよりも一段上の、言語などに関わるものということができる。

\section{メディアアートと呼ばれるモノとの交わり}\label{ux30e1ux30c7ux30a3ux30a2ux30a2ux30fcux30c8ux3068ux547cux3070ux308cux308bux30e2ux30ceux3068ux306eux4ea4ux308fux308a}

ここからはメディアアートと呼ばれるものとメディア考古学、そして本作品の関係についての考察を行いたい。\\
「メディアアートと呼ばれるもの」と書き続けているのは、「メディアアート」という言葉の定義が非常に曖昧であるからである。\\
現状メディアアートと言っても元来の意味であった「ニューメディアアート」の文脈を引き継ぐ、いわゆる新しい技術(例えば2016年に於いてはバーチャル・リアリティやディープラーニングなどはそれらに当てはまるだろう)をアート作品に用いるものや、日本の代表的なメディアアートの美術館であるNTTインターコミュニケーションセンターでの2016年の年間展示「オープン・スペース」のタイトルが「メディア・コンシャス」であるように表現に関わるテクノロジーや素材(メディウム)が現代においてどういう文化的意味を持ちうるかについて言及する傾向のものなど、幾つかの分類が可能なようにも見える。

しかしながら本論考では敢えてメディアアートの中のジャンルのを分類し、その中のどこに本作品が位置するか、のような言及の仕方は避けたいと考える。\\
なぜならそれら複数のジャンルに跨るものも当然存在するし、そのような線引きを行うことで見えなくなってしまう要素も存在すると考えるからである。

日本のメディアアート観に対する統一見解も未だまとまったものが存在するところではないが、馬定延「日本メディアアート史」など幾つかまとめようとする試みは出てきている。\\
しかし「日本メディアアート史」でのアプローチは、一般的な美術史の様に代表的な作品を列挙するのではなく、背景的に起こった出来事(例えば大阪万博など)、セゾンや草月アートセンター、NTTのICCなど舞台となった場を中心として、メディアアートとは何なのかを探るようなアプローチとなっている。\\
馬は序章においては

\begin{quote}
1970年代以降の日本のエレクトロニックアート紹介において、美術館における展覧会でなく、博覧会の名前が羅列されているのはなぜだろうか。なぜこれらの場を並べずには、日本のメディアアートの軌跡を語る事ができなかったのだろうか。
\end{quote}

\begin{quote}
こうした問題意識から、本書は個別の作家や作品ではなく、その背景をなす時代像に焦点を当ててみる。すなわち、本書はメディアアートの作品論と作家論を可能な限り排除して書かれたメディアアート史である。このような方法論が、究極的には、すべての表層的な要素、移り変わっていく背景を取り除いたあとに残る、メディアアートの本質たるものを強調することを目的にしていることは言うまでもない。
\end{quote}

またあとがきにおいては本書中で挙げられた作品写真は本文で言及されていないものばかりかつ、作品の説明などはなく最低限の情報しか載せていないことに触れた上でこう述べる。

\begin{quote}
筆者の明確な意図は、これらの画像を通して、何がメディアアートなのかではなく、むしろその定義不可能性を提示し、読者に問いかけることである。
\end{quote}

と、ある種明確にメディアアートの定義は不可能だと言い切っている。\\
実は本書は日本を代表するメディアアーティストである藤幡正樹についての研究から始まり、彼を中心として起きた出来事の歴史をまとめた論文が元になっていると本人も語っている。その為この本もかなり引いた目線で語っているとは言え日本のメディアアートの全てを包括して語っているとは言い難い。

しかしながら私はこの「引いた目線」、歴史を一つの観点で見定めようとせずに、起きた事象を冷静に見つめ、通底する物を引き出そうとする姿勢はメディア考古学的アプローチと強く結びつくところがあると考える。その為メディアアートの中での位置づけを考える際の大きな参考の一つとしたい。

何が「引いた目線」がメディア考古学的視点と結びつくとか、と言うのは特にフータモのメディア考古学観、特に彼の言う拡張された「トポス」概念である。

トポス概念について書きます

\section{メディアアートにおける本作品の立ち位置}\label{ux30e1ux30c7ux30a3ux30a2ux30a2ux30fcux30c8ux306bux304aux3051ux308bux672cux4f5cux54c1ux306eux7acbux3061ux4f4dux7f6e}

\chapter{まとめ}\label{ux307eux3068ux3081}

\section{今回作った作品の持ちうる意味}\label{ux4ecaux56deux4f5cux3063ux305fux4f5cux54c1ux306eux6301ux3061ux3046ux308bux610fux5473}

\section{今回の作品制作の行為の持ちうる意味}\label{ux4ecaux56deux306eux4f5cux54c1ux5236ux4f5cux306eux884cux70baux306eux6301ux3061ux3046ux308bux610fux5473}

\section{結び}\label{ux7d50ux3073}

\chapter*{参考文献}\label{ux53c2ux8003ux6587ux732e}
\addcontentsline{toc}{chapter}{参考文献}

\hypertarget{refs}{}
