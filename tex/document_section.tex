\chapter{序論}\label{ux5e8fux8ad6}

筆者は卒業制作として、音響遅延線メモリーというコンピューターの初期に使われていた記憶装置を題材にした音響装置作品の制作を行った。\\
音響遅延線という装置の概要については本論で述べることにするが、一連の制作においてはすでに淘汰され使われなくなったメディア装置、その中でも特に記憶装置を取り上げ、現代の技術を取り込み別のメディア装置として再生させるという作品制作のプロセスを取った。

\section{動機と問題意識}\label{ux52d5ux6a5fux3068ux554fux984cux610fux8b58}

今日、デジタルネイティブという言葉が存在するように我々が何か技術を使うときにアナログ、デジタルと言う境目を意識することは少なくなったように思える。また、それに伴い「保存する」ことと「通信する」事の境界は曖昧になりつつ有る。

例えば、MicrosoftのWordに代表される文書作成ソフトをWebブラウザ上で動作させるようにしたGoogle
Docs\footnote{Googleの提供する、ウェブブラウザ上で動作する文書作成ソフト。Wordなどの既存のフォーマットを扱うこともできる。元々2005年にUpstartle社のWritelyというWebサービスから始まり2006年にGoogleが買収しGoogle
  Documentとなった。共同編集機能はサービス開始当初からの特徴であった。\url{https://www.google.co.jp/intl/ja/docs/about/}}では書き込んだデータはリアルタイムでサーバに送信され自動で保存され、さらに共同編集をしている場合自分以外の書き込みも逐次反映されていく。\\
ここで、手紙の紙=保存メディアの発展形であったはずのGoogle
Docsはもはや手紙という行為そのもの=通信にもなっている。そしてこれらを使うとき、石版や紙のような媒体の物質性は意識されず、人間同士のやり取りだけが意識される。\\
このような時代においてそもそも情報を保存する、伝えるとは一体どういうことなのだろうか?

情報を保存するというのは、その前提に伝えるということがあって始まるのだが、通信技術の発達と計算機の性能向上によって、わざわざ一度保存せずとも、速く遠くに伝えることができるようになってしまった。それでは何を目的として何の情報を保存する必要があるのだろうか?

その疑問に迫るために筆者はコンピューター最初期に使われた「通信し続けることで保存する記憶装置」をである音響遅延線を可聴音を使い復元させるという行為を行った。

\section{メディアアートをどう考察するべきか?}\label{ux30e1ux30c7ux30a3ux30a2ux30a2ux30fcux30c8ux3092ux3069ux3046ux8003ux5bdfux3059ux308bux3079ux304dux304b}

筆者は「メディアアート」と呼ばれる領域に興味を持ち、自分の制作以外にもメディアアートと呼ばれるモノのサウンドプログラムの開発などを行なってきた。\\
そのため本論考においても「メディアアート」としての作品についての考察を行いたいと考えている。\\
すなわち「この作品はメディアアートの文脈の中でどういう位置づけにあるか?」という問いに答えなければいけないが、これには様々な困難がつきまとう。そもそもメディアアートとは何か?という問いに対して統一された見解のようなものは今も昔も存在しているとは言えないからである。\\
本論考に於いては日本におけるメディアアートの歴史をまとめる数少ない試みの内の一つである、2014年に発行された馬定延「日本メディアアート史」において用いられた「作品が作られる舞台となった施設やイベント、社会的状況を中心にメディアアートの歴史を考察する」というアプローチを参考にし、敢えて先行作品例との比較などでの文脈付けを行わずに考察することを試みた。

\section{メディア考古学という学問アプローチ}\label{ux30e1ux30c7ux30a3ux30a2ux8003ux53e4ux5b66ux3068ux3044ux3046ux5b66ux554fux30a2ux30d7ux30edux30fcux30c1}

先行作品例などではない考察をするための議論の軸として、一つのヒントとなる学問のアプローチが有る。今回の制作では「すでに淘汰されたメディア装置を研究し、別の形のメディア装置を作り上げる」というプロセスを取った。それに類似した、メディア考古学と呼ばれる学問のアプローチ法がある。メディア考古学とは、考古学という名前が付いているものの特定の学会が存在するわけでは無く、アプローチ法と書いたように学問のプロセスそのものを指す言葉であり、1980年代頃からいくつかこの名前がメディア研究において使われ始めた。メディア研究者のエルキ・フータモなどが現在における代表的な研究者である。その方法自体も具体的に確立されてるわけではなく、研究者ごとに微妙に異なっているのだが、日本のメディア研究者であり、フータモの著書を訳した太田純貴は

\begin{quote}
メディア考古学を最大限一般的にしたかたちで定義すれば、「日々増殖するメディアテクノロジーによって、埋没してしまったメディア文化やそれがもたらす経験についての言説の掘り起こし」であり、「大半のメディア考古学者たちに共通するのは、メディア文化についての規範的で正統的な物語を突き抜けて「掘り下げて」、省かれたものや的外れに終わった解釈を指摘すること」とまとめることができるだろう。
\end{quote}

と述べている。\\
フータモは論文内にて積極的にメディアアートと呼ばれるモノを取り上げる\autocite{huhtamo:mediaarcheology}。メディア考古学と呼ばれる言葉の存在する以前から、メディア考古学的アプローチを取り続けているアーティストが存在すると言うのだ。\\
。\\
本論考においては「送れ|遅れ/post|past」がメディア考古学的アプローチの制作と言えるのかどうかの考察及び、「日本メディアアート史」での考察法において、メディアアートを考察する上での視点についてもメディア考古学的アプローチと言うものが考えられるということを指摘し、最終的にメディアアート史の中での本作品の位置づけというものを行う。

\section{本論考の構成}\label{ux672cux8ad6ux8003ux306eux69cbux6210}

以上の前提を元に、本論考は、以下のように構成される.

\begin{enumerate}
\def\labelenumi{\arabic{enumi}.}
\tightlist
\item
  序論(本章)
\item
  音響遅延線という装置について
\item
  卒業制作「送れ|遅れ/post|past」について
\item
  制作手順と技術的解説
\item
  メディア考古学的メディアアート考察
\item
  メディア考古学的メディアアート制作
\item
  まとめ
\end{enumerate}

第2章においては、まず前提知識として卒業制作の題材となった装置である音響遅延線の技術解説と、それが作られた時代背景について解説する。\\
第3章において、卒業制作作品の全体的な解説を、3年次において制作した「Acoustic
Delay (⇔) Memory」との違いの比較を交え行う。\\
第4章においては、実際の制作の進行の記録と、作品のシステムとしての詳細な技術的解説及びそこに用いられた技術についての解説を行う。

第5章においては、「淘汰された装置を掘り起こす」という作品制作のプロセスについて、「メディア考古学的アプローチ」が作品制作とどう関わっていたかについて考察する。

第6章において、「メディア考古学的アプローチ」がメディアアートの歴史、文脈考察においてどのように関わるかを考察し、本作品がメディアアート史の中でどのような位置づけになるかについて考察したい。

第7章にて、メディアアート制作とメディアアート史を考える場合において「メディア考古学的アプローチ」がどのように適用できるのかを改めてまとめ、結びとする。

\chapter{音響遅延線という装置について}\label{ux97f3ux97ffux9045ux5ef6ux7ddaux3068ux3044ux3046ux88c5ux7f6eux306bux3064ux3044ux3066}

\section{概要}\label{ux6982ux8981}

音響遅延線メモリー(Acoustic Delay line
memory)という装置は1940\textasciitilde{}1960年頃の、初期電子計算機(いわゆる「コンピューター」)に使われていた記憶装置である。\\
(定義によって諸説あるのだが、一般的には)世界初のコンピュータと言われているENIACの次の世代として作られたアメリカのEDVAC(1951年)やイギリスのEDSAC(1949年)(EDSACはElectronic
Delay Storage Automatic
Calculatorの略であり、名前にも現れている)、また日本で最初の商用コンピュータであるFUJICでも\footnote{\section{技術解説}\label{ux6280ux8853ux89e3ux8aac}}使用されている。

開発者はENIACの開発にも携わっていたJohn Presper
Eckertであるが、はじめに導入予定であったEDVACでは特許関係で開発が遅れ、それより先に公開されていたEDVAC開発のための指針に影響を受けたEDSACが先に音響遅延線を採用したコンピュータとなった。

(適宜参考文献あげます)

音響遅延線メモリーは、水銀で満たしたタンクと、その一端に取り付けられたピエゾ素子のマイクロフォン、反対の端に取り付けられたやはりピエゾ素子のトランスデューサー(スピーカー)、及び増幅(波形整形)回路、入出力の制御回路から成り立っている。

時間を一定間隔で区切り、その間に超音波のパルスを発する/発しないを保存したい二進数データの1/0に対応させ、スピーカーから出力する。超音波のパルスは水銀中の音速、1451.4
m/sで伝わりマイクまで到達する。マイクで検出した信号を、回路の中で、パルスが存在すれば1、存在しなければ0という処理を施すと元送信した二進数データが復元される。それをもう一度同じスピーカーに送信すると、同じデータ列が循環し続けることとなる。最大で保存できるデータは区切る時間の間隔(データの送信レート)、マイクとスピーカー間の距離、音速を決定する温度に依存する。

音波の媒体として水銀を使っているのは送信、受信に使われるピエゾ素子との音響インピーダンスが近い事によって、マイクでの受信時に反射波が生じにくく、データにエラーが少ない、エネルギー効率がよい事が理由である。

記憶装置としての特徴は、逐次読み出し/書き出しであること、すなわち保存されているデータは任意のタイミングで読み出せるのではなく、一周循環してマイクの点に来るまで読み出せないことがある。\\
また、データの保存容量を増やすためにはパルスの間隔を短くすること、マイクとスピーカー間の距離を長くする、タンクを増やすのいずれかになる。そして読み出し、書き出しの速度もパルス間隔に依存し、計算自体に使う電気的な0/1の波形の速度よりは大幅に遅くならざるを得ず、仕組み上性能向上に限界が存在していた。\\
また、温度によって音速が変化してしまい、データの送信レートが変化してしまうので温度を一定に保つ仕組み(恒温槽)が必要になり、物理的に装置が巨大化せざるを得ないという欠点もあった\footnote{ただし日本のFUJICでは独自に恒温槽をなくし、読み出し速度を温度に依存して変化させることで逆転的に対応しシステムの簡略化を図っていた。}。

図とか入れる、適宜参考文献

\section{誕生から淘汰されるまでの時代の技術的移り変わり}\label{ux8a95ux751fux304bux3089ux6dd8ux6c70ux3055ux308cux308bux307eux3067ux306eux6642ux4ee3ux306eux6280ux8853ux7684ux79fbux308aux5909ux308fux308a}

EDVACに先行するENIACでは、回路を手でパッチを組みつなげることによりプログラムを構成していた。その為計算する問題を変える際は毎度パッチを組み替えないといけないなどの問題があった。

当時は真空管が不安定ですぐ焼ききれる→真空管でメモリーを組むより安価で安定\\
→磁気コアメモリの登場により台頭される、トランジスタの登場で完全に消滅

\section{技術的特異性}\label{ux6280ux8853ux7684ux7279ux7570ux6027}

音響遅延線メモリーは他の記憶装置と比べて大きな特徴を持っていると筆者は考えている。それは「通信し続けることで記憶装置としての機能を持つ」という点である。\\
記憶装置同士の比較や、カテゴリ分けには主に

\begin{enumerate}
\def\labelenumi{\arabic{enumi}.}
\tightlist
\item
  データのアクセス方法
\item
  読み出し時にデータを保持できるか(破壊/非破壊)
\item
  揮発/不揮発(電源が無くともデータを保持できるか)
\item
  書き換え可能か否か
\item
  容量
\end{enumerate}

などが主であるが、個別の装置を細かく見ていくとどれとも当てはまりづらい項目が幾つか出て来る。\\
例えばデータアクセス方法の例で考えてみる。\\
記憶装置のアクセス方式は大きく分けて

\begin{enumerate}
\def\labelenumi{\arabic{enumi}.}
\tightlist
\item
  直接アクセス/Random Access
\item
  逐次アクセス/Sequencial Access
\end{enumerate}

とされることが多い。

直接アクセスは現在使われているハードディスクやフラッシュメモリ、60年代に使われた磁気コアメモリなどが当てはまり、物理的に保存されているデータの位置に関わらず任意のデータを指定して読み出すことができる。\\
逐次アクセスは磁気テープやパンチカードのような媒体が当てはまる。はじめからデータを順番に読み出さないと任意のデータまでたどり着くことが出来ないタイプのものである。

ここで、ハードディスクドライブはデータのアクセス方法は一般的にはランダムアクセスに分類されるが、空き領域が連続している場合は連続した位置にデータを保存し、その場合読み出し法はシーケンシャルアクセスとなり読み出し速度が早くなる、といった微妙な例が出てくる。

またそもそも1962年の論文では

\begin{enumerate}
\def\labelenumi{\arabic{enumi}.}
\tightlist
\item
  循環アクセス/Cyclic Access
\end{enumerate}

という3つ目の分類が存在しており、音響遅延線メモリや磁気ドラムメモリのようなデータが常に循環し続けるものについてはこちらに分類されている(文献)。このように

さて、以上のようにそもそもデジタル記憶装置の分類自体が曖昧な部分が多数あることを理解した上で、敢えてわかりやすくカテゴリ分けしたときの音響遅延線メモリーの特徴は「揮発性かつシーケンシャルアクセスのみ」であることと言える。\\
この両性質を持ち合わせたメモリは現状存在せず、またこの両性質を同時に備えるのは「物質の状態を変化させ、固定する」ことを行わない音響遅延線メモリーの仕組み自体が大きく影響している。\\
もうちょっと書く。。。

\chapter{作品概要}\label{ux4f5cux54c1ux6982ux8981}

\section{前作品「Acoustic Delay (⇔)
Memory」について}\label{ux524dux4f5cux54c1acoustic-delay-memoryux306bux3064ux3044ux3066}

本作品は前年2015年に制作した「Acoustic Delay (⇔)
Memory」に続く要素が多く含まれるので、本作品の説明に入る前に前作品について説明する。

「Acoustic Delay (⇔)
Memory」は標準的なスピーカーとマイクロフォン、簡単な電子回路及びコンピュータを使用して構成された音響装置作品である。2015年12月12、13日、音楽環境創造科の制作・研究発表会、千住
Art Path 2015 において、東京芸術大学千住キャンパスの倉庫2で展示された。

回路はトランジスタでの標準的なロジックICのみを使用し1950年代に使用された音響遅延線とほぼ等価な回路を構成している。\\
それによりマイクとスピーカーの間の遅延時間で8bit(実際には、通信のためのスタートビットとストップビットをそのまま入れているため10bit)のデータを保存し続けている。\\
PCとは標準的なRS232の一般的なシリアル通信で読み出しと書き込み(上書き)ができるようになっている。展示中はWeb上にインターフェースが用意してあり、保存されているデータ列8bitのAsciiコードに対応した文字が表示される。右側には8つの長方形が並び、ビットの1/0が白・黒の色に対応して表示されている。\\
直接その部分を編集しキーボードで文字を打ち込み、Enterキーを押すか長方形をクリックするかでデータを書き込む事もできる。\\
これは直接USBで接続されているPCからだけでなく、同じURLを開けば携帯電話からでも世界中どのコンピュータからでもデータの読み/書きができる。\\
すなわちこの音響遅延線は一種のクラウドストレージのような機能を持つ。

また、もちろん保存されているデータは音として循環し続けているので、マイクとスピーカーの間に立って経路を遮ったり、手を叩くなど大きなノイズが発されればデータは変化してしまう。

(画像) 付録にキャプションのpdfとか付けます

\section{「送れ\textbar{}遅れ/post\textbar{}past」の解説}\label{ux9001ux308cux9045ux308cpostpastux306eux89e3ux8aac}

本作品の展示は2016年11月4、5、7、8、9日の5日間、東京芸術大学千住キャンパスの第7ホールにて行われた。

詳細な実装については次の項で述べるが、簡単に作品の概要を説明する。\\
本作品は図nに示すとおりの5台のコンピュータにより構成されるが、その役割は2台、2台、1台に分けられる。しかし個々のコンピュータは独立したプログラムによって制御され、LAN等のネットワークを介してお互いのコンピュータを制御することはない。

まず、PC1と2は2台で1つの音響遅延線としての機能を持つ。\\
PC1は1kHzに乗せてスピーカーから信号を送信し、PC2のマイクが1kHz上の信号を受信する。受信したデータと全く同じデータをPC2は4kHzに乗せてスピーカーから発信する。PC1はマイクから4kHz上の信号を受信し、また同じデータを1kHzに乗せて送り返す。\\
こうすると、ノイズや遮蔽物などが無い限り同じデータを保持し続ける。\\
PC2は更にデータの書き込み機能を備えている。PC2の画面の右側にはGoogle
Docsが開かれており、左側にはドキュメント上の文字が1つずつ区切られて表示されており、クリックすることで2つまで選択できる。画面下側にあるwriteボタンをクリックすると2つの文字をUTF-16でデコードしたビット列として上書きして送信する。

PC3は、PCの内蔵マイクでPC1の発する信号を読み取り、データに変化があったのを検知するとそれを2文字のUTF-16にエンコードし画面上に表示する。またそれがMac
OSの標準読み上げ機能で読み上げ可能だった場合読み上げる\footnote{実のところは、展示期間5日間中で文字を表示するようになったのは2日目から、文字が正しくデコードを始めたのは3日目午後からであった。それでもPC1で書き込んだデータをそのまま表示することは一度もなかった。これについては技術的解説において考察する。}。

PC4と5は1や2と同様にそれぞれ独立したプログラムが稼働していて、マウスやキーボードなどの操作が自動化されている。\\
PCにはGoogle
Docsで共通した書類が開かれている。まずPCの内蔵マイクで、一定の音量を検知すると、Google
Docs上の音声入力ボタンをクリックする。\\
音声区間が終了すると再び音声入力ボタンをクリックし、音声入力を終了する。その後、ドキュメント上の文章をすべて選択し、それをMac
OS上の標準読み上げ機能を使い読み上げる。\\
こうすることで、タイミングが合うとPC4が読み上げた文章をPC5で音声入力でドキュメント上に入力し、それをPC5が読み上げ再びPC4が音声入力し\ldots{}という状態が繰り返される。

\includegraphics[width=1.00000\textwidth]{img/postpast1.jpg}~

\section{比較と考察}\label{ux6bd4ux8f03ux3068ux8003ux5bdf}

「Acoustic Delay (⇔)
Memory」と「送れ\textbar{}遅れ/post\textbar{}past」における大きな違いは「装置の分離」にあると考える。\\
比較の単純さのために後者のPC1,2の部分を取り出して考えると、それぞれの部分はスピーカー、マイク、オーディオインターフェース\footnote{オーディオインターフェース}、PCが結線されており、PC1とPC2はハードウェア的に完全に分離されている。\\
つまり、個別の装置は受け取ったデータを送り返すだけの「通信装置」として機能する。\\
しかし2台の装置がそれぞれ同じデータを送り返し続けることで同じデータが保持され、結果的に「記憶装置」としての機能が成立する。\\
本来的に音響遅延線という装置も同じデータを受取り送信し続けることで記憶装置としての機能を持っていた。しかしそれは一台のハードウェアの中での通信であり、外見を見ればやはり1つの装置でしか無い。\\
しかし通信の繰り返しでデータの保存ができるという根本的な原理を考えると装置が2台以上に分離していても同じ機能を作ることが可能であり、その状態こそ通信と記憶の曖昧な状態を作り出すことが可能だと考え、2台に分けることにした。

\chapter{制作手順と技術的解説}\label{ux5236ux4f5cux624bux9806ux3068ux6280ux8853ux7684ux89e3ux8aac}

\section{実際の制作スケジュール}\label{ux5b9fux969bux306eux5236ux4f5cux30b9ux30b1ux30b8ux30e5ux30fcux30eb}

\begin{itemize}
\tightlist
\item
  6月:音響遅延線をテーマに作品制作を決定
\item
  7月:使える技術についての検討、無線通信技術や海中音響通信技術などの勉強
\item
  8月:適応フィルタ、低レイテンシーハードウェアの検証など
\item
  9月:千住キャンパス第7ホールで展示することを決定、音声入力と読み上げを使ったシステムも入れることを決める
\item
  10月:後半は第7ホールでの実地検証、読み上げシステムの実装
\item
  11月:展示
\end{itemize}

\section{制作に用いられた技術}\label{ux5236ux4f5cux306bux7528ux3044ux3089ux308cux305fux6280ux8853}

\subsection{音響遅延線メモリー部分}\label{ux97f3ux97ffux9045ux5ef6ux7ddaux30e1ux30e2ux30eaux30fcux90e8ux5206}

\subsubsection{信号処理のブロック・ダイヤグラム}\label{ux4fe1ux53f7ux51e6ux7406ux306eux30d6ux30edux30c3ux30afux30c0ux30a4ux30e4ux30b0ux30e9ux30e0}

イラレとかで書きます。。。

\subsubsection{位相偏移変調}\label{ux4f4dux76f8ux504fux79fbux5909ux8abf}

装置が2台に分かれるにあたって、従来の音響遅延線の仕組みをそのまま使おうとすると、マイクから受信した時に相手の信号と自分が発信した信号が混ざってしまうという問題点がある。今回はそれを2台の装置での使用周波数帯域を変更することで混線しないようにした。\\
周波数を変更するための変調方式には、載せる搬送波の振幅に信号を対応させる振幅偏移変調(ASK)、周波数変化に対応させる周波数偏移変調(FSK)、位相の変化に対応させる位相偏移変調(PSK)、振幅と位相を両方変化させる直行位相振幅変調(QAM)など様々な方式があるが、今回はQPSK、もしくは4QAMと呼ばれる、2bitのデータを位相の45°、135°、225°、315°の4状態に割り当てる方式を採用した。

(図)

図のように、基準となる位相から45°、135°、225°、315°がそれぞれデータの00、01、10、11に対応する。

\subsubsection{実際の実装方法}\label{ux5b9fux969bux306eux5b9fux88c5ux65b9ux6cd5}

主な実装には関数型音声処理言語のFAUST(Functional-AUdio-STream)を利用した。\\
FAUSTはフランスの音響研究機関、GRAMEの開発した言語であり、サンプル単位でのデジタル音声信号処理を非常に抽象的に記述することや、一度C++\footnote{C++について}にコンパイル(変換)されてからスタンドアロンアプリケーション、Max/MSP\footnote{Maxについて}やPuredata\footnote{Puredataについて}など様々な環境上で動作させることができるという特徴を持つ。

今回はほとんどの音声処理部分をFAUSTで記述し、Cycling'74
Max上でリアルタイムにコンパイルできるfaustgen\textasciitilde{}という環境を使用した。Maxは主に波形の確認等デバッグや、ブラウザとのPC内でのUDP通信のためのインターフェース用に使われた。\\
FAUSTを選択した理由としては構想当初の段階でスマートフォン上のブラウザや音声入出力の遅延の少ない小型コンピュータ(Bela\footnote{BeagleboneBlackについて})上で動作させることも想定しており、様々なプラットフォームで同じ音声処理を共通したコードで使いまわせるというメリットが大きかったためである。\\
また、Maxのような処理ごとのブロックをグラフィカルに繋いでプログラムを作るものはそれゆえに処理の抽象化に限度があり、例えば幾つかのパラメータを変更した同じ処理ブロックを直列につなぎ、何個つなぐかを簡単に変更したい、という時にはその都度ブロックをつなげ直すか、メタプログラミングのような回避策が必要となる。\\
その点においてFAUSTは並列、直列、入れ子構造など様々な接続法において接続数を変える時にパラメータの変更のみで済むという面がある。\\
結果的にはFAUSTはすべてMax上で動作させる形になったが、FAUSTで記述したコードは最適化された形でコンパイルされるため(正確なベンチマークなどの比較はしていないが)比較的低負荷で動作させることが可能であったなど、幾つかのメリットはあったと考える。

\subsubsection{使える技術と使えない技術}\label{ux4f7fux3048ux308bux6280ux8853ux3068ux4f7fux3048ux306aux3044ux6280ux8853}

今回、実装するに

適応フィルタに於けるマルチパス除去などをしようとすると遅延が生じることなど

\subsection{読み上げによる仮想遅延線メモリー}\label{ux8aadux307fux4e0aux3052ux306bux3088ux308bux4eeeux60f3ux9045ux5ef6ux7ddaux30e1ux30e2ux30eaux30fc}

PuredataとAppleScriptでの読み上げ機能\\
スクリーンショットとか

\chapter{メディア考古学的メディアアート史考察}\label{ux30e1ux30c7ux30a3ux30a2ux8003ux53e4ux5b66ux7684ux30e1ux30c7ux30a3ux30a2ux30a2ux30fcux30c8ux53f2ux8003ux5bdf}

本章においては、作品を考察するにあたってまずメディア考古学とは何か、そしてメディアアートと呼ばれるものの考察を如何になすべきかについてをメディア考古学的アプローチを手がかりに考えていく。

\section{メディア考古学とは}\label{ux30e1ux30c7ux30a3ux30a2ux8003ux53e4ux5b66ux3068ux306f}

序論でも述べたように、メディア考古学とは特定の学問領域を指すのではなく学問のアプローチである。\\
序論で引用した太田の定義に加えて共通する特徴を挙げるのであれば、メディアの歴史を語る際にはよく、何が技術的に新しかったかを挙げ、またそのテクノロジーが与えた文化への影響を考察する、と言った態度が取られる。これは裏返せば新しい技術が登場することによって新しい文化が作られていく、いわゆる技術決定論的な立場を暗黙的に取っている事が多い。\\
これに対してメディア考古学的アプローチを取るものは新しいメディア装置が生まれる際に何か文化からの暗黙的な要請によって新しくテクノロジーが生み出されることも含め、より技術と文化の発展の相互作用を注意深く精査する。また同時代での注目だけではなく、歴史上で文化的に同じような言及がされていたりしないかを調べたりなど、時代を横断することで歴史観を見直すこともある。

中心的な研究者として挙げられるのはエルキ・フータモ、ユシ―・パリッカ、ジークフリード・ツィーリンスキー、自分で名乗ることはないもののフータモらが引用することも多いフリードリヒ・キットラーやその教え子であるウォルフガング・エルンストなどが挙げられる。\\
正確にはアプローチが異なるというよりも、ツィーリンスキーが自分の研究を「anarcheology」(アナーキー考古学)と呼んだように、自らの領域を厳格に定義しない事自体がメディア考古学の特徴となっている。

今回筆者の作品の考察をするにあたっては、そのように微妙に異なるアプローチの中から、代表的な研究者であるエルキ・フータモと、聴覚文化(音響再生産、複製技術)への数少ないメディア考古学的アプローチの研究として挙げられている、ジョナサン・スターン「聞こえくる過去」\autocite{stern:audiblepast}のアプローチを主に参考とした。

両者の研究態度にももちろん違いは存在するので、簡単にその概要を述べる。\\
フータモは(後述するが)元々文学領域で使われる用語であった「トポス」概念を拡張し、様々な年代のメディアで現れる共通した文化的言及を探し出す。\\
例えば「妖精エンジンを分解する」と題した項で「機械のブラックボックスの中で働く``小人''という表現」を中心として「見えざる神の手」「その時背後では?」のようなメディア装置の広告だったり、風刺画の中などに現れる定形表現を見つけ出し、歴史に共通する人間のメディアに対する態度を表出させてるとも言えるし、ある共通した項目を抜き出した新しいメディア史の記述ということもできる。

スターンの記述は、「聞こえくる過去」に於いては「音響再生産技術(Sound
Reproduction
Technology)」、一般的にはレコードやテープなどの複製技術と呼ばれるものについての言及にとにかく集中しているものの、それをイヤー・フォノトグラフと呼ばれる時代的に音響再生産技術の始まる直前の物を取り上げたり、聴診器や、電信のような再生産以前の技術と交えて議論したり、レコードのような複製物と死に対する観念に集中して議論したり、とにかく多面的な考察がなされる。\\
もうちょっと書きます。。。

\section{メディアアートは如何に考察されるべきか?}\label{ux30e1ux30c7ux30a3ux30a2ux30a2ux30fcux30c8ux306fux5982ux4f55ux306bux8003ux5bdfux3055ux308cux308bux3079ux304dux304b}

ここからはメディアアートと呼ばれるものとメディア考古学、そして本作品の関係についての考察を行いたい。\\
「メディアアートと呼ばれるもの」と書き続けているのは、「メディアアート」という言葉の定義が非常に曖昧であるからである。\\
現状メディアアートと言っても元来の意味であった「ニューメディアアート」の文脈を引き継ぐ、いわゆる新しい技術(例えば2016年に於いてはバーチャル・リアリティやディープラーニングなどはそれらに当てはまるだろう)をアート作品に用いるものや、日本の代表的なメディアアートの美術館であるNTTインターコミュニケーションセンターでの年間展示「オープン・スペース」の2016年のタイトルが「メディア・コンシャス」であるように表現に関わるテクノロジーや素材(メディウム)が現代においてどういう文化的意味を持ちうるかについて言及する傾向のものなど、幾つかの分類が可能なようにも見える。

しかしながら本論考では敢えてメディアアートの作品群を分類し、その中のどこに本作品が位置するか、のような言及の仕方は避けたいと考える。\\
なぜならそれら複数のジャンルに跨るものも当然存在するし、そのような線引きを行うことで見えなくなってしまう要素も存在すると考えるからである。(あいまいなので変えたい)

日本のメディアアート観に対する統一見解も未だまとまったものが存在するところではないが、馬定延「日本メディアアート史」などで幾つかまとめようとする試みは出てきている。\\
しかし「日本メディアアート史」でのアプローチは、一般的な美術史の様に代表的な作品を列挙するのではなく、背景的に起こった出来事(例えば大阪万博など)、セゾンや草月アートセンター、NTTのICCなど舞台となった場を中心として、メディアアートとは何なのかを探るようなアプローチとなっている。\\
このようなアプローチを取った理由について馬は序章においては

\begin{quote}
1970年代以降の日本のエレクトロニックアート紹介において、美術館における展覧会でなく、博覧会の名前が羅列されているのはなぜだろうか。なぜこれらの場を並べずには、日本のメディアアートの軌跡を語る事ができなかったのだろうか。
\end{quote}

\begin{quote}
こうした問題意識から、本書は個別の作家や作品ではなく、その背景をなす時代像に焦点を当ててみる。すなわち、本書はメディアアートの作品論と作家論を可能な限り排除して書かれたメディアアート史である。このような方法論が、究極的には、すべての表層的な要素、移り変わっていく背景を取り除いたあとに残る、メディアアートの本質たるものを強調することを目的にしていることは言うまでもない。
\end{quote}

と述べているし、またあとがきにおいては本書中で挙げられた作品写真は本文で言及されていないものばかりかつ、作品の説明などはなく最低限の情報しか載せていないことに触れた上でこう述べる。

\begin{quote}
筆者の明確な意図は、これらの画像を通して、何がメディアアートなのかではなく、むしろその定義不可能性を提示し、読者に問いかけることである。
\end{quote}

と、ある種明確にメディアアートの定義は不可能だと言い切っている。\\
実は本書は日本を代表するメディアアーティストである藤幡正樹についての研究から始まり、彼を中心として起きた出来事の歴史をまとめた論文が元になっていると本人も語っている\footnote{2015/12/12\textasciitilde{}13の千住
  Art Path
  2015(東京芸術大学千住キャンパス制作・研究発表展)での毛利嘉孝とのトークにおいて。またあとがきの最後でも遠回しであるが日本で研究する目的が藤幡について研究したかったから、ということを述べている。}。その為この本もかなり引いた目線で語っているとは言え日本のメディアアートの全てを包括して語っているとは言い難い。

しかしながら私はこの「引いた目線」、歴史を一つの観点で見定めようとせずに、起きた事象を冷静に見つめ、通底する物を引き出そうとする姿勢は「メディア考古学的アプローチ」と強く結びつくところがあると考える。その為本作品のメディアアートの中での位置づけを考える際の大きな参考の一つとしたい。

何が「引いた目線」がメディア考古学的視点と結びつくか、と言うのは特にフータモのメディア考古学観、特に彼の言う拡張された「トポス」概念である。

トポス概念とはもともと文学の中の常套句のようなものであり、メディア考古学を考察するにあたってフータモはこの概念を拡張している。\\
例えばバーチャル・リアリティにおける「没入感」などの言葉や「複雑な仕組みの機械の様子を表現するのに妖精や小人がマシンを動かしている比喩」などがある。

要するに、歴史は現代の状況から記録を見返して作られるものであり、現代の状況が歴史観の形成に与える影響は大きい。例えばメディア史であれば技術の与えた影響が文化を変えて、また文化の要請が新たな技術を生み出すことで変化し続けることと同様に、メディアアートは時代を取り巻く技術及び文化の状況が変化することでその都度歴史観が大きくひっくり返される可能性があるのではないだろうか。

それに加えて、従来のアート作品の考察において無意識にとられる作家が影響を受けた作品群の列挙、

\section{本論考における作品考察のアプローチ}\label{ux672cux8ad6ux8003ux306bux304aux3051ux308bux4f5cux54c1ux8003ux5bdfux306eux30a2ux30d7ux30edux30fcux30c1}

次章において卒業制作作品についての考察を行うが、以上のような理由から参考作品群の分類や、作家たちの系譜を辿りその中に自分の作品を位置づけるのではなく、あくまで作品に使われている技術や装置について、その装置や技術の生まれた社会的背景、現代の状況との関係が作品のコンセプトや作品が現在持つ意味についてを考察していきたい。

\chapter{メディア考古学的メディアアート制作}\label{ux30e1ux30c7ux30a3ux30a2ux8003ux53e4ux5b66ux7684ux30e1ux30c7ux30a3ux30a2ux30a2ux30fcux30c8ux5236ux4f5c}

本章と次章においては文章の主体が「卒業制作を作ったアーティストとしての立場」の場合と、「メディア考古学の研究をする立場」、あるいは「メディアアートについて研究する立場」として俯瞰して自分の作品、あるいはそれを取り巻く環境について考察する場合とを混在して書かざるを得ないので、立場が変わる場合は明示して表すことにする。

\section{デジタル記憶装置はメディア考古学で掘れるのか}\label{ux30c7ux30b8ux30bfux30ebux8a18ux61b6ux88c5ux7f6eux306fux30e1ux30c7ux30a3ux30a2ux8003ux53e4ux5b66ux3067ux6398ux308cux308bux306eux304b}

\subsection{デジタル記憶装置に対する文化的言及}\label{ux30c7ux30b8ux30bfux30ebux8a18ux61b6ux88c5ux7f6eux306bux5bfeux3059ux308bux6587ux5316ux7684ux8a00ux53ca}

しかし、本作品の制作アプローチを二者のアプローチと比較しようとした時に大きな違いが出てくる。

そもそもは取り上げる装置の違いに起因する。音響遅延線だけではなく「デジタル記憶装置」という分類のものには「文化的な言及や現象」が存在していない事が多い。

その理由は主に2つである。

\begin{enumerate}
\def\labelenumi{\arabic{enumi}.}
\tightlist
\item
  記憶装置とは感性に対して直接作用するメディアでないこと。
\item
  デジタル記憶装置は扱う内容が極めて抽象化(エンコード)されること。
\end{enumerate}

「メディア考古学 ―過去と未来の対話のために―」の中で取り上げられる装置の実例を挙げると、万華鏡、ピンボールマシン、ゾートロープ(岩井俊雄の作品)、フォノグラフ(ポール・デマリニスの作品)などが挙げられる。\\
ピンボールマシンはともかく、それ以外の三者はすべて視覚や聴覚など、五感に対して何らかの作用を持つ装置であることは確かである。\\
しかし、デジタル記憶装置というものはデータをすべて一度0/1の並びに変換してその並びを保存する。\\
つまり保存するデータは言葉でも良いし、音楽でも画像でも、なんでもよい。\\
そのため、ハードディスクやフラッシュメモリ、そしてもちろん音響遅延線もレコードであったり写真のような文化を構成する一部としては考察し難い。

例えば最初は記録メディアがカセットテープであったウォークマンのことを考えてみると、1999年にフラッシュメモリ式のネットワークウォークマンと呼ばれるシリーズが発売され始める。\\
初期ウォークマンをカセットテープに付随する文化史の中に並べることはできるかもしれないが、フラッシュメモリにおいても同じ事が言えるのだろうか?\\
筆者自身はこれは不可能だと考える。その理由は先ほどから述べているように単にフラッシュメモリという記録メディアの保存できるものの種類が広範であることが大きい。\\
フラッシュメモリ型ウォークマンの与えた文化的影響についての考察はできるかもしれないが、フラッシュメモリ単体でそれが与えた文化的影響を考察するのは散漫にならざるをえない。\\
このように、デジタルメディアはその汎用性が故に記録、再生装置、もしくは記録/再生する内容とセットにしなければ考察する対象にならないのである。

それに加えて、音響遅延線メモリーが動作していたのは初期のコンピュータなので、フラッシュメモリのように広く一般に普及したわけでもなく、各マシンによって詳細な構造もそれぞれ異なるという点もある。

\subsection{音響遅延線メモリーの歴史の読み替え}\label{ux97f3ux97ffux9045ux5ef6ux7ddaux30e1ux30e2ux30eaux30fcux306eux6b74ux53f2ux306eux8aadux307fux66ffux3048}

とは言ったものの、少しでも音響遅延線に対するメディア史に関わる言及が存在したかどうかの調査の実践は行うべきだと考え実行した。その結果をここに書き記す。\\
唯一、聴覚文化との関わりがありそうな言及として見つかったのが、開発者であるジョン・プレスパー・エッカートへのインタビュー(1988年、スミソニアン協会により、国立アメリカ歴史博物館で行われた)の中で音響遅延線メモリーの仕組みについて解説している場面である。これは「送れ\textbar{}遅れ/post\textbar{}past」の展示の際、会場入り口で配布したインストラクションとなる文章にも抜粋して掲載した。

\begin{quote}
エッカート:例えば、水銀で満たされたタンクがあるとしよう。1000発のパルス音を片方の端から入れて、反対側の端から漏れ出てしまう前までは、それはメモリーと呼べるだろう。\\
問題は1ミリ秒後のことだ。もしパルスが1マイクロ秒ごとに一つづつタンクから出ていってしまえば、それでお終いだ。しかしここでこのパルスを毎回取り出して、もう一度元の形に整形してもう一度戻したとしたらどうだろう。パルスは言ったとおりに循環し続け同じ状態を保つ。
\end{quote}

\begin{quote}
インタビュアー: ということはその記憶装置は水銀の動きで保存していると?
\end{quote}

\begin{quote}
エッカート:
水銀を伝わる波によって保存しているんだ。水銀は動かずそのままだが、弾性波がその中を通り抜けていく。水銀の分子の前後に動き続ける。そして波はその中を伝搬していく。これをどうやって思いついたと思う?小さい頃のことを思い出そう。買い物に行く時に、母親が私にあれやこれを4\textasciitilde{}5個買ってきて欲しいと頼んでくる。それを書いてメモするのではなく、多分同じような小さい子供はみんなそうすると思うが、母親に送り出されてから店につくまでの道のり中ずっとその5つのことを自分の中で繰り返し自分に言い聞かせる。\\
こうすると若い私の短期記憶は店につく頃には長期記憶になっている。同じ原理だ。\\
音響遅延線も水銀のタンクの中に詰めて電気回路を通して戻して循環させている。私たちはこのタイプのメモリーを使って最初のUNIVAC、UNIVACⅠ{[}\^{}UNIVAC{]}を作り上げた。
\end{quote}

\autocite{eckertinterview}

{[}\^{}UNIVAC{]} UNIVACについて。

また、1953年時点のエッカート自身による記憶装置のシステムについてまとめたサーベイでも似たような記述が見られる。

\begin{quote}
遅延線メモリーの基本的仕組みはシンプルだ。情報のパターンが遅延媒体の中に挿入される。遅延媒体の終点から最初の場所へパターンを増幅回路とタイミング調整回路を通した上で引き戻してやると、パターンの循環の閉ループが成立する。簡単に説明すると、遅延線メモリーの仕組みは誰かに電話をかける時に、電話帳にあった電話番号をダイアルを回すまで頭のなかで繰り返し続けているときの短期記憶と等しい。
\end{quote}

\autocite{eckert1953survey}

このように、エッカートは音響遅延線の仕組み自体を人間の行為の比喩として説明している。

実際に音響遅延線を開発される際にこのアイデアや例えが先にあった上で開発されたのか、それとも開発後にわかり易く説明するためにこの例えを思いついたのかははっきりしていない。

音響遅延線の歴史については2章であらかた説明したが、そもそも音響遅延性ははじめからメモリーとして開発されたわけではなくはじめはレーダーに映り込むノイズ(移動しない物体)を取り除くために、単純な信号遅延処理のデバイスとして用いられ、その研究に関わっていたエッカートがパルスを戻すことで記憶装置としての機能を持たせていた。

個人的にはその転換点、どこで遅延装置を記憶装置として使用するに至ったかに興味があったのだが決定的な言及には最終的にたどり着けなかった。

そして結果的にこの言及が作品の制作上のシステムに影響することはなかったと考える。\\
それよりはGoogleドキュメントの音声入力のシステムであったりとか、現代の状況のほうが強く影響を与えていたような気がする。

〜つづく〜\\
結局、「送れ|遅れ/post|past」において、筆者は「手段は問わないデジタル装置」は逆説的に「何らかの伝達と認識の手段の存在」によってその機能が保証されていて、「その装置の機能の持つ手段から文化的な意味を''勝手''に読み出してより見える形にする」ということをしている、そしてその文化的意味が表出するためには「装置が現在の文化においては役に立たない」ことが必要となる

ということを丁寧に書きたい

\section{メディア考古学的アプローチでのメディアアート
-城一裕らの「車輪の再発明プロジェクト」から}\label{ux30e1ux30c7ux30a3ux30a2ux8003ux53e4ux5b66ux7684ux30a2ux30d7ux30edux30fcux30c1ux3067ux306eux30e1ux30c7ux30a3ux30a2ux30a2ux30fcux30c8--ux57ceux4e00ux88d5ux3089ux306eux8ecaux8f2aux306eux518dux767aux660eux30d7ux30edux30b8ux30a7ux30afux30c8ux304bux3089}

フータモらの挙げるようなメディアアーティスト(ポール・デマリニスや岩井俊雄ら)は自らのアプローチをメディア考古学と言うわけではなく\footnote{ただしポールデマリニスは1997年のNTTICCでの展示は「-Archaeology
  of Media-」というタイトルを付けている。}、フータモら研究者が考察する際に挙げているだけである。

それに比べて数少ない、作品(作品でなくとも、何かしらの「制作」プロジェクト)において明示的に「メディア考古学」の名前を挙げているプロジェクトとして、2015年にNTTのICCで行われたメディアアートを取り上げる通年展示「オープン・スペース2015」で展示された、情報科学芸術大学院大学(通称IAMAS)\footnote{日本の岐阜県大垣にあるメディアアートを専門にした大学院。}内のプロジェクト「車輪の再発明プロジェクト」がある。THE
SINE WAVE
ORCHESTRAのメンバーかつ研究者でもある城一裕を中心に、クワクボリョウタ、瀬川晃、松井茂ら講師陣と学生によって行われたプロジェクトである。

プロジェクトは以下のような文章で説明されている。

\begin{quote}
このプロジェクトでは実践を通じて歴史を読み替え,ありえたかもしれない「今」をつくりだします.淘汰されてしまった過去のメディアを再考察することで新しい現在のメディアへの理解を深める「メディア考古学」を足がかりに,視聴覚メディアを中心とした様々なメディアの形づくられてきた過程を調べ,その機能や役割が歴史的に固定される前の可能性について理解を深めます.\\
そして,コンピュータやネットワークを取り入れた個を主体としたものづくりの潮流である「パーソナル・ファブリケーション」以降の技術・社会環境において,この理解を芸術表現に活用する方法を,多様な作品制作の下地となる技法として提案します.\\
例えば,聴覚メディアの古典的な存在ともいえるレコード,その機能や役割を,画面上で描いた波形を溝として盤面に直接刻み込むという〈技法〉を用いて制作したいくつかの〈作品〉,を通じて再考するように,マス・メディアにおいては顕在化しなかったメディアの可能性を「再発明」します.\\
\autocite{iamas:RIWP}
\end{quote}

このプロジェクトでは、

\begin{quote}
城一裕《月の光に---エドアード・レオン・スコットとモホイ=ナジ・ラースローヘ---》
\end{quote}

\begin{quote}
1860/1923/2014年
\end{quote}

\begin{quote}
技法:予め吹きこまれた音響のない(もしくはある)レコード
\end{quote}

の様に、幾つかの「技法」たちが―レコードの他の例では「針穴をあけた紙を通したRGB光源による網点プロジェクション」、「写植文字盤による多光源植字」や「(1)コイル、(2)磁石、を与えられたものとせよ」のようにタイトルが付けられ、それぞれの技法を用いて作られた「作品」がタイトルを付けて展示される。

車輪の再発明プロジェクトにおいて挙げられる特徴はデジタルファブリケーション、パーソナル・ファブリケーション\footnote{デジタルファブリケーション}など個人の使えるモノづくりのための技術として新しいものを導入することで、あり得たかもしれない歴史を描き出すという、歴史の読み替えである。

本作品の制作に於いて、筆者は前作「Acoustic Delay (⇔)
Memory」制作後の考察をしている段階でこのプロジェクトについて知り、「メディア考古学」という概念の存在を知った。そして卒業制作で改めて音響遅延線を題材に作品を作る段階では「メディア考古学的アプローチ」というものには自覚的に制作を行っていた。そのためこのプロジェクトが本作品の制作に与えた影響は大きいと考えている。

特に思想的な影響としては「淘汰された装置を現代の技術を用いて復元させ、あり得たかもしれない今を作り出す」という点である。\\
特に現代で音響遅延線メモリーを作ろうとしても水銀は有毒性が強いため使えない、その為現代の状況で音響遅延線作ったら否が応でも異なる物が出来上がるだろうという考えであったり、その時代には存在しなかった無線通信技術の導入をする、などのアイデアはこのプロジェクトからの影響が強いと考える。

しかし、前項で説明した様に取り上げる装置がデジタル記憶装置であったため、結果的にこの作品は「車輪の再発明プロジェクト」とも違う道筋を辿らざるを得なかった。\\
車輪の再発明プロジェクトでは取り上げた装置、「レコード」ならばそれを「聴覚に作用する装置」として再発明している。

今回の作品に於いては「記憶装置」を「記憶装置」として再発明しているものの、実のところ重点を置いていたのは記憶装置としての機能ではなく記憶装置が成立するためのプロセスは何なのか?という点であった。\\
プロセスと言っても実際の記憶装置の仕組みが個別にどうなっているか、ではなく記憶装置の仕組みを勝手に人間の行動に置き換え、それが現代の技術ではどういった仕組みに再び置き換えられるのか?ということである。

研究者としての立場に戻ると、でもある意味これって技術決定論の方に寄るのでは?

アーティストとしての立場では、「仕組みだけを残したゾンビ状態にした」とか音声入力・読み上げの方では現在使われている最新の文化装置の組み合わせで同じことができるというものの提示による比較である。という点で技術の変化で何が起きたか?と言うよりは未整理のまま育ってきた文化と技術の中に同じ現象が発生する(若干トポス的)

\chapter{まとめ}\label{ux307eux3068ux3081}

ここまで、卒業制作「送れ|遅れ/post|past」について、メディア考古学的アプローチでの作品制作に対する考察と、メディア考古学的アプローチからメディアアート史に対して作品を位置づけることについての考察を行ってきた。

制作においては本作品はフータモやスターンの挙げてきたようなメディア考古学的アプローチとも、

\section{結び}\label{ux7d50ux3073}
